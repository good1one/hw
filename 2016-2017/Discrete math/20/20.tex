\documentclass[a4paper,11pt]{article}
\usepackage[T1]{fontenc}
\usepackage[utf8]{inputenc}
\usepackage{graphicx}
\usepackage{xcolor}


\usepackage{amsmath,amssymb,amsthm,textcomp}
\usepackage{enumerate}
\usepackage{multicol}
\usepackage{tikz}
\usepackage[english, russian]{babel}

\usepackage{geometry}
\geometry{total={210mm,297mm},
	left=25mm,right=25mm,%
	bindingoffset=0mm, top=20mm,bottom=20mm}


% custom footers and headers
\usepackage{fancyhdr}
\pagestyle{fancy}
\lhead{}
\chead{}
\rhead{Егерев Артем}
\lfoot{}
\cfoot{}
\rfoot{\thepage}
\renewcommand{\headrulewidth}{0pt}
\renewcommand{\footrulewidth}{0pt}

%%%----------%%%----------%%%----------%%%----------%%%

% перечислимость -  множество конструктивных объектов (например, натуральных чисел), все элементы которого могут быть получены с помощью некоторого алгоритма.
% вычислимость - функция называется вычислимой, если она вычисляется некоторым алгоритмом
% разрешимость - существует алгоритм, который, получив на вход любое натуральное число, через конечное число шагов завершается и определяет, принадлежит ли оно данному множеству


\begin{document}
\textbf{\large Задача 1}
\medskip\hrule\medskip
\textit{Верно ли, что множество наборов подряд идущих цифр длины 5, входящих  в десятичную запись числа $ \pi $ а) перечислимо  б) разрешимо? Можно считать, что любая цифра числа $ \pi $ вычислима.}  \\ 

\textsl{a)} Да, в качестве искомого алгоритма мы просто будем идти последовательно и выписывать все 5-ки цифр из числа $ \pi $ \\ 

\textsl{б)} Множество всех пятерок цифр из числа $ \pi $ - есть подмножество в конечном множестве натуральных чисел от 0 до $ 10^5 \Rightarrow $ данное множество будет являться разрешимым
\\ \\ \\


\textbf{\large Задача 2}
\medskip\hrule\medskip
\textit{Пусть множество $ X $ натуральных чисел перечислимо. Перечислимо ли множетсво $ Y \subseteq X $ тех чисел, сумма цифр которых равна 10?}  \\  

Обозначим за $ A $ - алгоритм, который выписывает все числа из множества $ X $. Нам известно, что мы можем вычислять сумму цифр конечного натурального числа за конечное число шагов. Поэтому, предъявив в качестве искомого алгоритма, функцию которая просто пробегает по всем числам из $ X $, сравнивает сумму его цифр с 10 и выписывает число в случае положительного ответа $ \Rightarrow Y  $ - перечислимо. \\
Заметим, что таким образом мы переберем все такие числа так как $ A $  является вычислимой, а значит на $ k $-ом шаге мы сможем добраться до любого $ x $. 
\\ \\ \\


\textbf{\large Задача 3}
\medskip\hrule\medskip
\textit{Докажите, что если $ A, B $ - перечислимые множества, то и множество $ A\times B $ перечислимо.} \\

Так как $ A, B $ являются перечислимыми, то существуют алгоритмы, выписывающие их элементы в строчку. Приведем следующий  алгоритм перечисления элементов декартового произведения $ A \times B $. Для каждого $ k $ от 1 до $ \infty $ будем выписывать пары $ (A_k, B_i), (A_i, B_k) $ для всех $ i $ от 1 до $ k $. Это работа аналогичная тому, что мы делали на семинаре и достаточно очевидно, что таким путем мы выпишем все элементы из $ A \times B $.
\newpage


\textbf{\large Задача 4}
\medskip\hrule\medskip
\textit{Всюду определенная $ f : \mathbb{N} \times \mathbb{N}$ строго возрастает и множество ее значений содержит все натуральные числа за исключением конечного множества. Докажите, что $  f $ - вычислима.} \\

Будем доказывать индукцией по $ f(0) $.  База индукции это когда разрывы отсутствуют вообще, тогда  $ f(x) = f(0) + x $ - вычислима. Теперь шаг: будем идти по промежутку, такому, что $ f(x + 1) = f(x) + 1 $ начиная с $ 0 $ и до максимального $ x_{max} $ (до разрыва). На этом промежутке наша функция вычислима и равна: $ f(x) = a + x $. Введем новую функцию $ g(x) = f(x_{max} + 1 + x) $. Так как она является подфунукцией функции, описанной в нашем задание, то она и сама обладает данными свойствами и из предположения индукции является вычислимой. \\
\[
	f(x) = 
	\begin{cases}
		f(0) + x \quad &x \leq x_{max} \\
		g(x) \quad &x > x_{max} 	
	\end{cases}
\]
Здесь стоит так же отметить, что размер любого разрыва конечен, исходя из условия задания.
\\ \\ \\


\textbf{\large Задача 5}
\medskip\hrule\medskip
\textit{Существуют ли такие множества $ X, Y \subseteq \mathbb{N}$, что $ X $ - разрешимо, $ X \cup Y $ разрешимо, a $ Y $ не разрешимо } \\ 

Да, существует. Возьмем за $ X $ -  множетсво натуральных чисел, $ Y $ - его неразрешимое подмножество (а такое существует, исходя из примера на семинаре). Тогда $ X = X \cup Y = \mathbb{N} $ -  разрешимо, $ Y $ - не разрешимо.
\\ \\ \\



\textbf{\large Задача 6}
\medskip\hrule\medskip
\textit{Пусть $ S $ - разрешимое множество натуральных чисел. Множество $  D $ состоит из всех простых делителей множетсва $ S $. Верно ли, что $ D $ перечислимо?}  \\

Для конечного натурального числа все его делители находятся за конечное число действий. Проверка на простоту для каждого из делителей осуществляется так же за конечное число действия. Таким образом, мы просто будем брать каждый элемент из $ S $, находить все его простые делители и проверять не записали ли мы уже это число в $ D $ $ \Rightarrow D $ - перечислимо.
\\ \\ \\


\textbf{\large Задача 7}
\medskip\hrule\medskip
\textit{Пусть $ f $ - вычислимая биекция между $ \mathbb{{N}} $ и $ \mathbb{N} $. Докажите, что обратная функци я $ f^{-1} $ тоже вычислима} \\ 

Для вычисления $ f^{-1}(y) $ просто будем идти по всем $ f(x) $ для $ x $ начиная с 1 и сравнивать это значение с $ y $. Совершая за каждый шаг конечное число шагов, получаем вычислимый алгоритм.  \\
При этом стоит отметить, что $ f^{-1} $ однознчачно определена, так как $ f $ -  биекция.
\end{document}










