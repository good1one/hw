\documentclass[12pt,a4paper]{scrartcl}
\usepackage[utf8]{inputenc}
\usepackage{amsmath}
\usepackage{amsfonts}
\usepackage{amssymb}
\usepackage{graphicx}
\usepackage[english, russian]{babel}

\title{}
\date{Егерев Артем, БПМИ-167}
\author{Домашняя работа по дискретной математике 15}

\begin{document}

\maketitle
\noindent
Теорема 1. В бесконечном множестве можем выделить счетное подмножество.\\
Теорема 2. Конечное или счетное объединение счетных мнжеств счетно.\\
Теорема 3. Если множество A счетно и B  $\subseteq$ A, то B конечно или счетно. \\
Теорема 4. Любое подмножество счетного - конечно или счетно. \\ 
Теорема 5. Объединение счетного и конечного - счетно. $\mathbb N \cup n \sim \mathbb N$
\\ \\
\noindent \textbf{Задача 1.}
\textit{Решение.} Верно.
Обозначим за С - счетное подмножетво в $A\setminus B$ (Теорема 1) \\
$$
A 
= (A \setminus B) \cup (A \cap B)  
=  (A \setminus B \setminus C) \cup C \cup (A \cap B) \sim
$$

\noindent Первый случай - A $\cap$  B - конечно: 

$$
\sim (A \setminus B \setminus C) \cup \mathbb N \cup n
\sim (A \setminus B \setminus C) \cup \mathbb N
\sim (A \setminus B \setminus C) \cup C
= A \setminus B 
$$
По теореме 5. \\

\noindent Второй случай - A $\cap$  B - счетно (Теорема 4) . Очень похоже: 
$$ 
\sim (A \setminus B \setminus C) \cup \mathbb N \cup \mathbb N
\sim (A \setminus B \setminus C) \cup \mathbb N
\sim (A \setminus B \setminus C) \cup C
= A \setminus B
$$
По теореме 2. \\
\\ \\ \\

\noindent \textbf{Задача 2.}
\textit{Решение.} Неверно. Возьмем A = B = $\mathbb N$ - натуральные числа $\Rightarrow A \bigtriangleup B = \varnothing  \nsim A$
\\ \\ \\


\noindent \textbf{Задача 3.}
\textit{Решение.} Верно. Это частный случай задачи 1, когда $A \cap B$ - конечно.
\\ \\ \\


\noindent \textbf{Задача 4.}
\textit{Решение.} 
Между любыми двумя различными действительными числами существует рациональное. \\
Пробежав по всем интервалом, выберем произвольно по одному рациональному числу из каждого. \\
Множество рациональных - счетно $\Rightarrow$ его подмножество (получившиеся числа) - счетно или конечно.
\\ \\ \\


\noindent \textbf{Задача 5.}
\textit{Решение.} Шаг: есть бесконечное множество B.\\ 
Выделим в нем  счетное  подмножество A: \\
Работая с нумерацией в счетном A, разделим все  на четные  (Множество $A_1$) и нечетные элементы (Множество $A_2$) . \\
$$A_1 \cap A_2 = \varnothing \Rightarrow \forall A_3 \subseteq A_2,  A_1 \cap A_3 = \varnothing $$
$A_2$ -  счетно, а значит бесконечно. $A_1$ не будет пересекаться ни с какими подмоножествами $A_2$. Отсюда, мы  бесконечное множество разбили на   $B \setminus A$,  бесконечное $A_2$  и  счетное $A_1$, которое не будет пересекаться с дальнейшими  подмножествами в $B \setminus A_1$ $\Rightarrow$ за один шаг можем всегда увеличивать количество непересекающихся счетных подмножеств на один $\Rightarrow$ их количество бесконечно. \\
\\ \\ \\


\noindent \textbf{Задача 6.}
\textit{Решение.} Для задания переодической функции  достаточно знать $f(0), f(1), ... \\ f(t-1)$ при фиксированном $t$. \\
$f(i)$ можем выбрать  любое из $\mathbb Z$ $\Rightarrow$ количество таких функций  $\mathbb Z^{t}$ - счетно, так как $\mathbb Z$ счетно (Теорема 2) \\
$t \subseteq \mathbb N$ $\Rightarrow$ количетсво периодических функций есть  счетное объединение счетных множеств, а оно счетно (Теорема 2 )
\\ \\ \\


\noindent \textbf{Задача 7.}
\textit{Решение.} Рассмотрим следующую функцию для  произвольной конечной последовательности натуральных чисел:
$$
f(S) = 
\begin{cases}
1:  \varnothing , \text{если } S =  \varnothing 
\\
2:  S,  \text{если длина S} = 1
\\
3:  \text{Для остальных - в качестве первого элемента берем первый элемент S, } 
\\ 
\text{каждый последующий это (разность -  1)  между предыдущем} 
\\
\text{и текущем в возрастающей последовательсти H}
\\
\text{(сместили на 1 для того, чтобы мы могли  работать с нулями) } 
\end{cases}
$$
Пример: $f(\text{1 0 2 3}) =$ 1 (1 + (0 + 1))  (2 + (2 + 1))  (5 + (3 + 1)) = 1 2 5 9 \\
 Заметим, что $f \text{ и } f^{-1}$ будут функциональны и инъективны, а значит $f$ - биекция. 
\end{document}