\documentclass[11pt,a4paper]{scrartcl}
\usepackage[utf8]{inputenc}
\usepackage{float}
\usepackage{amsmath}
\usepackage{amsfonts}
\usepackage{amssymb}
\usepackage{graphicx}
\usepackage{color}
\usepackage[english, russian]{babel}


\begin{document}
\begin{flushright}
	Егерев Артем \\
	Домашнее задание №19\\
\end{flushright}
\textbf{\Large Задача 1}
\medskip\hrule\medskip
\textsl{Постройте схему полиномилального размера для функции $ f: \{0, 1\}^{\binom{n}{2}} \rightarrow \{0, 1\}$, равной единице, тогда и только тогда, когда в данном на вход графе есть изолированные вершины.} \\ \\
\fbox{\textit{Решение:}} \\ \\

\noindent Пронумеруем наши вершины числами от 1 до $ n $. Обозначим за $ v_{ij} \in \{0, 1\} $ ребро между вершинами $i, j$. \\
Тогда проверка изолированности для $ i $-ой вершины это просто:
\begin{gather*}
	\overline{\bigvee_{j \in 1 ... n, \text{ }j \neq i } v_{ij}}
\end{gather*}   
\noindent Применив дизъюнкцию к  $ n $ таким выражениям (так как количество вершин $ n $ соответственно) получаем ответ:
\begin{gather*}
	\bigvee_{i \in 1... n} \big ( \overline{\bigvee_{j \in 1 ... n, \text{ }j \neq i } v_{ij}} \big )
\end{gather*}
Оценим теперь размер полученной схемы. Для каждой из $ n $ вершин мы добавляем не более $ n $ элементов, отсюда размер схемы есть $ O(n^2) $.
\newpage








\textbf{\Large Задача 2}
\medskip\hrule\medskip
\textsl{Треугольником на графе называется тройка вершин, попарно соединенных между собой. Постройте схему полиномиального размера, для функции $ f: \{0, 1\}^{\binom{n}{2}} \rightarrow \{0, 1\}$ равной единице, тогда и только тогда, когда в данном на вход графе нет треугольков} \\ \\
\fbox{\textit{Решение:}} \\ \\

\noindent Пронумеруем наши вершины числами от 1 до $ n $. Обозначим за $ v_{ij} \in \{0, 1\} $ ребро между вершинами $i, j$. \\
Тогда проверка на отсутствие треугольника для вершин $ i, j, k $ это просто:
\begin{gather*}
	\overline{v_{ij} \land v_{ik} \land v_{kj}}
\end{gather*}  
Применив глобальную конъюнкцию  ко всем элементам получаем ответ на нашу задачу:
\begin{gather*}
	\bigwedge_{i, j, k \in 1... n, i \neq j, i \neq k, j \neq k} \overline{v_{ij} \land v_{ik} \land v_{kj}}
\end{gather*}
Оценим теперь размер полученной схемы. Заметим, что количество элементов в глобальной конъюнкции - $  \binom{n}{3} $, остальные элементы дают вклад не более  чем $ O(n) $, отсюда общий размер схемы $ O(n^3) $.
\newpage









\textbf{\Large Задача 3}
\medskip\hrule\medskip
\textsl{Постройте схему полиномиального размера  для функции $ f: \{0, 1\}^{\binom{n}{2}} \rightarrow \{0, 1\}$, равной единицы тогда и тогда данный на вход граф связен и содержит эйлеров цикл.} \\

\fbox{\textit{Решение:}} \\ \\
\noindent Проверка на связность осуществляется возведением соотвествующей матрицы смежности в степень $ n $. Перемножение матриц осуществляется за $  O(n^3) \Rightarrow $ размер такой проверки $ O(n^4) $.  \\ \\
Граф содержит эйлеров цикл тогда и только тогда, когда он связен и степень каждой его вершины четна. \\
Проверка того, что степень вершины  $ i $ четна:
\begin{gather*}
	\neg (v_{i1} \oplus v_{i2} \oplus ... \oplus v_{in} \textsl{ // для всех кроме $ v_{ii}$ //})
\end{gather*}
Примечание: $ \oplus - XOR$ можно представить с помощью конъюнкций и дизъюнкций за константу. \\ 
Для получения ответа нам нужно взять $ n $ конъюнкций, для каждой вершины соответственно.  \\
Оценим теперь размер полученной схемы. Проверка на связность - $ O(n^4) $, проверка на степени - $ n \cdot O(n) = O(n^2) \Rightarrow $ итоговый размер $ O(n^4) $ 
\newpage




















\textbf{\Large Задача 4}
\medskip\hrule\medskip
\textsl{Докажите, что любую монотонную функцию  от  $ n $ переменных можно вычислить схемой размера $ O(n2^n)$, используя только конъюнкцию и дизъюнкцию.} \\

\fbox{\textit{Решение:}} \\ \\
Пусть монотонная функция начинает принимать значение 1, когда в наборе становиться $ i $ единиц. \\
Тогда :
\begin{gather*}
	f = \bigvee (x_{u(1)} \land x_{u(2)} ... \land x_{u(n)}), \qquad \textit{по всем $ u $ - неупорядоченным наборам $ i $ чисел из $ n $}
\end{gather*}
Всего таких слагаемых $ \binom{n}{i} < 2^n$, в каждом не более $ n $ слагаемых. \\
Отсюда общий размер схемы не превосходит $ O(n) \cdot O(2^n) = O(n2^n)$
\newpage















\textbf{\Large Задача 5}
\medskip\hrule\medskip
\textsl{Докажите, что существует функция от $ n $ переменных ($ n > 2 $), не вычисляющаяся в базисе $ \{\oplus, \cdot, 1 \}$, схемой размера $ n^{100} $. } \\ \\

\fbox{\textit{Решение:}} \\ \\
Данный нам базис есть базис в многочлене Жегалкина, в котором любая функция имеет однозначное представление. \\
Воспользумся функцией из следующего номера:
\begin{gather*}
	f(x_1 ... x_n) = 1 \oplus x_1 \oplus ... \oplus x_n \oplus x_1x_2 \oplus ... \oplus x_1x_2...x_n
\end{gather*}
Очевидно, что на вычисление одного слагаемого мы тратим хотя бы 1 элемент (так как каждое слагаемое в сумме у нас уникальное). Всего слагаемых - $ 2^n $, отсюда минимум нам понадобится $ 2^n $ элементов, что в ассимтотике дает больше чем $ n^{100} $.
\newpage




















\textbf{\Large Задача 6}
\medskip\hrule\medskip
\textsl{Докажите, что в базисе $ \{ \oplus, \cdot, 1 \} $ любая функция от $ n $ переменных вычисляется схемой размера не более $ 2^{n + 1} $.} \\ \\
\fbox{\textit{Решение:}} \\ \\
Данный нам базис есть базис в многочлене Жегалкина, в котором любая функция представима в виде:
\begin{gather*}
	f(x_1 ... x_n) = a_0 \oplus (a_{11} x_1) \oplus (a_{12} x_2) ... \oplus (a_n x_1 ... x_n)
\end{gather*} 
где каждая $ a $ - это булева константа. \\
Рассмотрим худший случай, когда нам нужно посчитать максимальное количество операций. Достаточно очевидно, что тогда все $ a $  должны  быть равны 1. \\ \\
Тогда количество слагаемых для сложения по модулю 2 - количество всевозможных подмножеств есть $ 2^n $, отсюда количество $ \oplus $ - $ (2^n - 1) $ \\ \\
При вычисление каждого из слагаемых количество элементов будет увеличиваться на 1, так как на начальном этапе оно действительно увеличивается на  1 (при подсчете $ x_i $), а потом каждое слагаемое получается из слагаемого, длины на один меньше и одного из $ x $. Поэтому на вычисление всех слагаемых мы потратим так же не более $ 2^n $. \\
Итого, в суммарном размере имеем $ (2^n - 1) + 2^n < 2^{n + 1}$.
\newpage



















\textbf{\Large Задача 7}
\medskip\hrule\medskip
\textsl{Булева функция $ f: \{0, 1\}^n \rightarrow \{0, 1\}$ называется линейной, если она представима в виде:} 
\begin{align*}
	f(x_1, ..., x_n) = a_0 \oplus (a_1 \land x_1) \oplus ... \oplus (a_n \land x_n)
\end{align*}
\textsl{Для некоторого набора $ (a_1, ... a_n) \in \{0, 1\}^n$ булевых коэффицентов.} \\
\textsl{Докажите, что схема, использующая только линейные функции, вычисляет линейную функцию.} \\ \\ 

\fbox{\textit{Решение:}} \\ \\
Сложная функция $ f $ представляется как композиция простых.\\
Отсюда нам достаточно доказать, что сумма по модулю 2 и логическое умножение на константу сохраняет линейность:
\begin{gather*}
	f(x_1, ..., x_n) =
	f_1(x_1, ..., x_n) \oplus ... \oplus f_n(x_1, ..., x_n) = \\
	= a_{10} \oplus (a_{10} \land x_1) \oplus ... \oplus (a_{1n} \land x_n) \oplus 
	... 
	a_{n0} \oplus (a_{n1} \land x_1) \oplus ... \oplus (a_{nn} \land x_n) = \\
	= (a_{10} \oplus ... \oplus a_{n0}) \oplus (a_{11} \oplus ... \oplus a_{1n}) \land x_1 \oplus ... \oplus (a_{n1} \oplus ... \oplus a_{nn}) \land x_n = \\
	= a'_0 \oplus (a'_1 \land x_1) \oplus ... \oplus (a'_n \land x_n) - \textsl{ линейная функция}
\end{gather*}
где $ a'_i =  a_{i0} \oplus ... \oplus a_{i0}$
\begin{gather*}
	f(x_1, ..., x_n) \land b = (a_0 \oplus (a_1 \land x_1) \oplus ... \oplus (a_n \land x_n)) \land b = \\
	= (a_0 \land b) \oplus ((a_1 \land b) \land x_1) \oplus ... \oplus ((a_n \land b) \land x_n) = \\
	a'_0 \oplus (a'_1 \land x_1) \oplus ... \oplus (a'_n \land x_n) - \textsl{ линейная функция}
\end{gather*}
где $ a'_i =  a_i \land b$
\newpage


















\textbf{\Large Задача 8}
\medskip\hrule\medskip
\textsl{Докажите, если $ f(x_1, ..., x_n) $ - нелинейная функция, то конъюнкция $ x_1 \land x_2 $ вычислсяется схемой в базисе $ \{0, 1, \neg, f \} $} \\ \\

\fbox{\textit{Решение:}} \\ \\
Так как $ f(x_1, ..., x_n) $ является нелинейной функцией, то в ее разложение в виде полинома Жегалкина присутствует конънкция  двух $ x_i, x_j $. Для удобства, переобозначим  их за $ x_1, x_2 $.
\begin{gather*}
	f(x_1 ... x_n) = a_0 \oplus (a_{11} x_1) \oplus (a_{12} x_2) ... \oplus (a_n x_1 ... x_n) = \\ =  x_1x_2 f_1(x_3, ... x_n) \oplus (x_1 \land f_2(x_3, ... x_n)) \oplus (x_2 \land f_3(x_3, ... x_n)) \oplus f_4(x_3, ... x_n)
\end{gather*}
Так как $ f_1 $ не может оказаться нулевой $ \Rightarrow $ найдется такой набор $ (x'_3, ... x'_n) $, что $ f_1 = 1$. \\
Обозначим  $ f_2(x'_3, ... x'_n) $ за $ a_2 $,  $ f_3(x'_3, ... x'_n) $ за $ a_3 $, $ f_(x'_3, ... x'_n) $ за $ a_4 $. \\
Введем следующую функцию :
\begin{gather*}
	r(x_1, x_2) = f(x_1 \oplus a_3, x_2 \oplus a_2, x'_3, ... x'_n) = \\
	=  (x_1 \oplus a_3)(x_2 \oplus a_2) \oplus x_1 a_2 \oplus x_2 a_3 \oplus a_4 = \\
	= x_1x_2 \oplus 2x_1a_2 \oplus 2x_2a_3 \oplus (a_2a_3 \oplus a_4) =
	x_1x_2 \oplus (a_2a_3 \oplus a_4) =  
\end{gather*} 
Итого имеем:
\[	
	x_1 \land x_2 = 
	\begin{cases}
		r \qquad &\textit{при } a_2a_3 \oplus a_4 = 0 	\\
		\neg r \qquad &\textit{при } a_2a_3 \oplus a_4 = 1 	\\
	\end{cases}
\]
 















\end{document}