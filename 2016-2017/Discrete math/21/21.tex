\documentclass[a4paper,11pt]{article}
\usepackage[T1]{fontenc}
\usepackage[utf8]{inputenc}
\usepackage{graphicx}
\usepackage{xcolor}


\usepackage{amsmath,amssymb,amsthm,textcomp}
\usepackage{enumerate}
\usepackage{multicol}
\usepackage{tikz}
\usepackage[english, russian]{babel}

\usepackage{geometry}
\geometry{total={210mm,297mm},
	left=25mm,right=25mm,%
	bindingoffset=0mm, top=20mm,bottom=20mm}


% custom footers and headers
\usepackage{fancyhdr}
\pagestyle{fancy}
\lhead{}
\chead{}
\rhead{Егерев Артем}
\lfoot{}
\cfoot{}
\rfoot{\fbox{\thepage}}
\renewcommand{\headrulewidth}{0pt}
\renewcommand{\footrulewidth}{0pt}

%%%----------%%%----------%%%----------%%%----------%%%

\begin{document}
\textbf{\large Задача 1}
\medskip\hrule\medskip
\textit{Докажите, что для любой универсальной функции $ U $ множество $ \{U(p, p) \; : \; p \in \mathbb{N} \} $ совпадает с $ \mathbb{N} $.}  \\[16pt]
Для каждого $ n \in \mathbb{N} $ рассмотрим функцию  $ f(x) = c $, для которой найдется такое число $ p : U(p, x) = f(x) = c $ (просто по определению универсальной функции). Тогда и $ U(p, p) = c $. \\[3pt]
То есть у нас имеется инъекция в натуральные, но и $ p \in \mathbb{N} $, отсюда значений функций  не  более чем $ \mathbb{N} $, значит их ровно $ \mathbb{N} $. \\ \\ \\


%%%----------%%%----------%%%----------%%%----------%%%


\textbf{\large Задача 2}
\medskip\hrule\medskip
\textit{Верно ли, что для любой универсальной вычислимой функции $ U $ множество}
\begin{align*}
\textit{а) } \{p \; | \; U(p, p^2) определено\} \\
\text{б) } \{p \; | \; U(p^2, p) определено\}
\end{align*}
\textit{неразрешимо?} \\ \\

\textsl{а)} Будем решать от противного $ \Rightarrow $  допустим оно разрешимо. Тогда найдется всюду определенная $ f $, такая что:
\begin{align*}
f(p) = 
\begin{cases}
1 \quad & p - \text{ полный квадрат, }U(\sqrt{p}, p)  \text{ определено и равно 0}\\
0 \quad & p - \text{ полный квадрат, }U(\sqrt{p}, p)  \text{ определено и не равно 0}\\
-1 \quad & \text{  иначе }
\end{cases}
\end{align*}
Так как $ U $ - Увф $ \Rightarrow \exists p \; : \; U(p, x) = f(x) $. $ f(x) $ является всюду определенной, отсюда и $ f(p^2) = U(p, p^2) - $ определено. 
\begin{align*}
\begin{cases}
U(p, p^2) = 0 \Rightarrow f(p^2) = 1 \neq U(p, p^2) \\
U(p, p^2) = 1 \Rightarrow f(p^2) = 0 \neq U(p, p^2)
\end{cases}
\Rightarrow \text{ противоречие}
\end{align*}

\textsl{б) } Неверно. Построим контрпример. 
\begin{gather*}
U_0(p, x) = 
\begin{cases}
p \quad &p - \text{полный квадрат} \\
U(num(p), x) \quad &\text{иначе}
\end{cases}
\end{gather*}
Здесь $ U $ - произвольная вычислимая функция, $ num - $ номер числа в последовательности натуральных чисел без квадратов. Стоит так же отметить, что все эти функции являются вычислимыми. \\[2pt]
Получаем, что рассматриваемое в условие задачи множество есть просто множество натуральных, которое, в свою очередь $ - $ разрешимо.
\newpage



%%%----------%%%----------%%%----------%%%----------%%%


\textbf{\large Задача 3}
\medskip\hrule\medskip
\textit{Докажите, что во всяком бесконечном разрешимом множетсве натуральных чисел есть перечислимое неразрешимое подмножество.} \\ \\

Так как данное нам множество является бесконечным и  разрешимым среди натуральных, то мы можем перейти к биекции этого множества с натуральными, сопоставив каждому числу его индекс в отсортированном множестве. \\
Аналогично рассуждениям из прошлой задачи $ B = {x : U(x, x) определено} $ неразрешимо. Кратко:
\begin{gather*}
f(p) = 
\begin{cases}
1 \quad & U(p, p) \text{ определено и равно 0} \\
0 \quad & U(p, p) \text{ определено и не равно 0} \\
-1 \quad & \text{ иначе }
\end{cases}
\quad
\begin{cases}
U(p, p) = 0 \Rightarrow f(p) = 1 \neq U(p, p) \\
U(p, p) = 1 \Rightarrow f(p) = 0 \neq U(p, p)
\end{cases}
\end{gather*}
Рассмотрим произвольное бесконечное подмножество натуральных $- \; A $. Тогда $ f_1 $ и $ f_2 $ - характеристические функции этих подмножеств:
\begin{gather*}
f_1(x) = 
\begin{cases}
1 \quad &x \in A \\
0 \quad &x \text{ иначе }
\end{cases}
\qquad
f_2(x) = 
\begin{cases}
1 \quad &x \in B \\
0 \quad &x \text{ иначе }
\end{cases}  
\end{gather*}
$ C = A \cap B $. Характеристическая функция $ - f_3(x) = f_1(x) \cdot f_2(x) $
невычислима, так как $ f_2(x) $ невычислима. Отсюда, мы нашли такое подмножество.
\\ \\ \\



%%%----------%%%----------%%%----------%%%----------%%%


\textbf{\large Задача 4}
\medskip\hrule\medskip
\textit{Докажите, что бесконечное подмножество $ \mathbb{N} $  разрешимо тогда и только тогда, когда оно является областью значений всюду определенной возрастающей вычислимой функции из $ \mathbb{N} $ в $ \mathbb{N} $}  \\ \\
$ \rightarrow $ Так как подмножество разрешимо, то мы просто можем пойти по всем натульным числа и проверять принадлижит ли данное число подмножеству. Если да, то будем выписывать это число. Очевидно, что при таком подходе будет выписана бесконечная последовательность возрастающих чисел.  \\[2pt]
$ \leftarrow $  Сопоставим следующий алгоритм, проверяющий принадлежность $ x $ подмножеству. Бежим по всем $ i $ от 1 до $ x $ и сравниваем значения возрастающей функции с $ x $. Если встретилось $ -  $ выводим 1, нет - 0. Так как  функция возрастающая, то достаточно очевидно, что число $  x $ после шага $ x $ встретиться уже не может. 
\newpage


%%%----------%%%----------%%%----------%%%----------%%%


\textbf{\large Задача 5}
\medskip\hrule\medskip
\textit{Докажите, что любое бесконечное перечислимое множество сожержит бесконечное перечислимое подмножество.} \\ \\ 
 
Определим подмножество как возрастающую последовательность чисел, которые выводит перечислитель множества. Другими словами, перечислитель выдает нам число, мы сравниваем его с предидущем, уже выписанным, и, если оно больше, то выписываем его. \\[2pt]
Таким образом мы выпишем бесконечно много возрастающих чисел (так как все числа конечные) и, пользуясь предыдущей задачей, получаем, что данное подмножество является вычислимым, а значит и перечислимым.
\\ \\ \\




%%%----------%%%----------%%%----------%%%----------%%%


\textbf{\large Задача 6}
\medskip\hrule\medskip
\textit{Докажите, что перечислимо множество программ, которые останавливаются хотя бы на одном входе. Более формально: пусть $ U - $ универсальная вычислимая функция, а $ S - $ множество тех $ p $, для которых $ U(p, x) $ определена хотя бы при одном $ x $. Тогда $ S -$ перечислимо.} \\ \\ 
Отметим тот факт, что область определения $ U $ есть  просто множество пар из двух чисел, которое так же обладает свойством перечислимости. \\[2pt]
Искомое множество - есть просто первая координата этих пар, а значит оно так же перечислимо.   
\end{document}










