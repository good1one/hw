\documentclass[12pt,a4paper]{scrartcl}
\usepackage[utf8]{inputenc}
\usepackage{amsmath}
\usepackage{amsfonts}
\usepackage{amssymb}
\usepackage{graphicx}
\usepackage[english, russian]{babel}

\title{Домашняя работа по дискретной математике 14}
\author{Егерев Артем, БПМИ-167}


\begin{document}

\maketitle

\noindent \textbf{1.}
Доказательство.\\
Пусть было продано $N$ билетов. Тогда общий бюджет лотереи 100$N$, общая сумма выиграша - $100 \cdot p \cdot N = 40N$. Обозначим сумму выиграша i-человека за $\alpha_i$. Тогда:

$$
E[\alpha] 
=\sum\limits_{i = 1}^N \alpha_i p_i 
= \sum\limits_{i = 1}^N \frac{\alpha_i}{N} = 
\frac{100\cdot p \cdot N }{N} 
= 100 \cdot 0.40 
= 40
$$
По неравенству Маркова:

$$
Pr[\alpha \geq 5000] \leq \frac{{E[\alpha]}}{5000} 
\Rightarrow  Pr[\alpha \geq 5000] \leq \frac{40}{5000} 
= 0.8\% < 1\%
$$
\\Доказано.
\\ \\ \\ \\




\noindent \textbf{2.}
Решение.\\
Обозначим за $N$ - суммарное количесво человек, $\alpha$ - продолжительность жизни произвольного человека

$$
E[ \alpha] 
= \sum_{k = 1}^{N} \alpha_k p_i 
= \sum_{i = 1}^{N/2} \alpha_i p_i +\sum_{j = N/2 + 1}^{N} \alpha_j p_j 
= \frac{1}{N} \sum_{i = 1}^{pN} \alpha_i +\frac{1}{N} \sum_{j = (1 - p)N + 1}^{N} \alpha_j 
=
$$

$$
= \frac{1}{N} \sum_{i = 1}^{N/2} \alpha_i +\frac{1}{N} \sum_{j = N/2 + 1}^{N} \alpha_j  
= \frac{1}{2} \cdot \frac{1}{N/2} \sum_{i = 1}^{N/2} \alpha_i + \frac{1}{2} \cdot \frac{1}{N/2} \sum_{j = N/2 + 1}^{N} \alpha_j 
=
$$

$$
= \frac{1}{2} \cdot E[\alpha \leq 8] + \frac{1}{2} \cdot E[\alpha > 8] = 26 
$$
Отсюда:

$$
E[\alpha \leq 8] + E[\alpha > 8] 
= 52
$$
Оценим  $E[\alpha \leq 8]$:

$$
1 \leq E[\alpha \leq 8] \leq 8
$$
Нижняя граница достигается по причине того, что каждый живет как минимум  один год,
верхняя - когда нет людей, живших  менее 8 лет.\\

$$
E[ \alpha > 8] \geq E[\alpha \geq 8] \Rightarrow E[\alpha \geq 8] \leq 52 - 1 
= 51
$$
Нижняя граница интервала равна 26. Становиться равной в случае, когда у нас нет людей, живших  менее 8 лет: мат ожидание  всех будет как раз совпадать с нашем  искомым  ответом.\\
Стоит отметить, что изменяя количество восьмерок, мы можем  плавно менять искомую величину, поэтому все значения интервала будут приемлемы .
\\Ответ: [26, 51] 
\\ \\ \\ \\




\noindent \textbf{3.}
Решение.\\
б) Обозначим за $p_i$  вероятность выпадения числа $i$ . Тогда:

$$
E_1 
= 1^2 \cdot p_1 + 2^2 \cdot p_2+ ... + 6^2 \cdot p_6 
= \sum_{i = 1}^{6} p_i \cdot i^2
$$

$$
E_2 
= \sum_{i = 1}^{6} \sum_{i = 1}^{6} i \cdot j \cdot p_i p_j  
$$
Рассмотрим:

$$
E_1 - E_2 =
 \sum_{i = 1}^{6} \sum_{i = 1}^{6} i \cdot j \cdot p_i p_j - \sum_{i = 1}^{6} p_i \cdot i^2 
= \sum_{i = 1}^{6} p_i i (\sum_{j = 1}^{6} p_j j  - i) 
= \sum_{i = 1}^{6} p_i i (\sum_{j = 1,  j \neq i}^{6} p_j j - i(1 - p_i)) 
= 
$$
Переход делается из расчета $\sum\limits_{i = 1}^{6} p_i = 1$

$$ 
= \sum_{i = 1}^{6} p_i i (\sum_{j = 1,  j \neq i}^{6} p_j j - i\sum_{j = 1,  j \neq i}^{6} p_j) = \sum_{i = 1}^{6} p_i i (\sum_{j = 1,  j \neq i}^{6} p_j (j - i)) 
= \sum_{i = 1}^{6} \sum_{j = 1,  j \neq i}^{6} p_i i \cdot p_j \cdot (j - i) 
=
$$
Следущий переход делаем отдельно, разбирая случаи  i < j и i > j (тогда можем поменять i и j местами, для того чтобы учесть соответствующее слагаемое в сумме).

$$
\sum_{1 \leq i < j \leq 6}^{}(ip_i (j - i) p_j + jp_j (i - j)p_i) 
= \sum_{1 \leq i < j \leq 6}^{}p_i p_j (j - i)(i - j) 
= -\sum_{1 \leq i < j \leq 6}^{}p_i p_j (j - i)^2 < 0
$$
Отсюда $E_2 > E_1$\\
Так как мы решали в общем случае, то и в а) это так же выполняется.
\\ Ответ:  $E_2 > E_1$ 
\\ \\ \\ \\



\noindent \textbf{4.}
Решение.\\
Обозначим за n - количество вхождений подслов ab в наше слово
.
$$
E[n] 
= \frac{\sum\limits_{i = 1}^{20 - 1} k_i}{N}
$$
Где N - количество всех слов ,$k_i$ - количество вхождений  ab на i-ой позиции (всего позиций 20 - 1 = 19)

$$ N = 2^{20} $$
Когда на i-ой позиции стоит ab, то у нас остается 18 свободных мест, на которые мы можем поставить a или b, отсюда $k_i = 2^{18}$

$$
E[n] 
= \frac{19 \cdot 2^{18}}{2^{20}} 
= \frac{19}{4}
$$
\\ Ответ: $\frac{19}{4}$.
\\ \\ \\ \\



\noindent \textbf{5.}
Решение.\\
В общем случае:

$$
E[k] = \frac{\sum\limits_{i = 1}^{10} i \cdot f(i)}{10^{15}}
$$
где $f(i)$ - количество различных последовательностей завтраков, в которых ровно i различных\\
По факту $f(i)$ - количество упорядеченных разбиений 15 дней на i различных  завтраков. \\
Вспомним, что S(15, i) - число Стирлинга - количество неупорядоченных изменений. Так же нужно еще выбрать сами номера различных завтраков - это $C_{10}^i$\\
Отсюда $f(i) = S(15, i) \cdot i! \cdot C_{10}^i = S(15, i) \cdot A_{10}^i$ 
\\ Ответ: $E[k] = \frac{\sum\limits_{i = 1}^{10} i \cdot S(15, i) \cdot A_{10}^i}{10^{15}}$.
\\ \\ \\ \\



\noindent \textbf{6.}
Решение.\\
Так как все перестановки различные, можем рассмотреть пару: перестанвку $\pi$ и ей зеркально отраженную $\pi^{-1}$ (мы можем разбить все перестановики на пары, так как их количество n! - четное, при всех n > 1). Рассмотрим произвольные числа a, b, стоящие на разных позициях в нашей перестановке. Заметим, что суммарно в $\pi$ и в $\pi^{-1}$ эти числа образуют ровно одну инверсию. Значит, общее количество инверсий в этих двух перестановках равно количесту всевозможных пар (так как мы выбирали числа произвольно) и равно $\frac{n(n-1)}{2}$

$$
E[I(\pi)] = \sum_{i = 1}^{n!} p_i \cdot I[\pi_i] = \frac{1}{2} \sum_{i = 1}^{n!} p_i (I[\pi_i] + I[\pi^{-1}_i]) = \frac{n(n-1)}{4} \sum_{i = 1}^{n!} p_i = \frac{n(n-1)}{4}
$$
При n = 1, $E[I(\pi)] = 0$.
\\ Ответ: $\frac{n(n-1)}{4}$.
\\ \\ \\ \\



\noindent \textbf{7.}
Доказательство.\\

$$
Pr[X \geq 6] \Leftrightarrow Pr[2^{X} \geq 64]
$$ \\
По неравенству Маркова:\\

$$
Pr[2^{X} \geq 64] \leq \frac{5}{64} < \frac{1}{10}
$$
Доказано.
\\ \\ \\ \\



\noindent \textbf{8.}
Решение.\\
Обознаим за $S_k$ - среднее количество жвачек, которое нужно купить после того, когда у нас уже есть k разных к, так чтобы последняя была новой. Тогда на этом шаге вероятнотсь вытащить новую - $\frac{n-k}{n}$, неоригинальную - $\frac{k}{n}$. Рассмотрим выбор следующий жвачки: если мы выбрали новую - то это подходит под наше условие, если нет, то нам потребуется еще  $S_k$ шагов, поэтому: 

$$
S_k 
= \frac{n-k}{n} + \frac{k}{n} (S_k + 1)
$$
Отсюда:

$$
S_k 
= \frac{n}{n - k}
$$
Получаем, что среднее число жвачек, которое необходимо купить для полной коллекции вкладышей, равн
о:
$$
S 
= S_0 + S_1 + ... + S_{n-1} = n(1 + 1/2 + 1/3 + ... + 1/n)
$$
\\Ответ: n(1 + 1/2 + 1/3 + ... + 1/n).
\\ \\ \\ \\
\end{document}