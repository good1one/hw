\documentclass[a4paper,11pt]{article}
\usepackage[T1]{fontenc}
\usepackage[utf8]{inputenc}
\usepackage{graphicx}
\usepackage{xcolor}


\usepackage{amsmath,amssymb,amsthm,textcomp}
\usepackage{enumerate}
\usepackage{multicol}
\usepackage{tikz}
\usepackage[english, russian]{babel}

\usepackage{geometry}
\geometry{total={210mm,297mm},
	left=25mm,right=25mm,%
	bindingoffset=0mm, top=20mm,bottom=20mm}


% custom footers and headers
\usepackage{fancyhdr}
\pagestyle{fancy}
\lhead{}
\chead{}
\rhead{Егерев Артем}
\lfoot{}
\cfoot{}
\rfoot{\fbox{\thepage}}
\renewcommand{\headrulewidth}{0pt}
\renewcommand{\footrulewidth}{0pt}

%%%----------%%%----------%%%----------%%%----------%%%

\begin{document}
\textbf{\large Задача 1}
\medskip\hrule\medskip
\textit{Пусть $ U(p, x) -$ главная универсальная вычислимая функция. Докажите, что найдется бесконечно много таких $ p $, что $ U(p, x) = 2017 $ для какого-то $ x $.} \\ \\

Для решения рассмотрим следующую всюду определенную функцию:
\begin{align*}
f_n(x) = 
\begin{cases}
2017 \quad &\text{ при } $ x = 0 $ \\
1 \quad &\text{ при } $ x = n $ \\
0 \quad &\text{ иначе}
\end{cases}
\end{align*}
где $ n - $ какое-то натуральное число. Так как $ \mathbb{N} - $ бесконечно и для любой функции $ f_n $ найдется свое $ p $ (просто по определению универсальной функции), то найдется бесконечно много $ p \; : \; U(p, 0) = 1017$. И да, данные $ p $ различны, так как функции различаются. 
\\ \\ \\





%%%----------%%%----------%%%----------%%%----------%%%



\textbf{\large Задача 2}
\medskip\hrule\medskip
\textit{Пусть $ U(p, x) - $ главная универсальная ывчислительная функция. Докажите, что найдется такое $ n $, что $ U(n, x) = nx $ для всех $ x $.} \\ \\

Рассмотрим функцию $ V(x, y) = xy $. Так как $ U - $ главная, то существует функция $ s(x) $, такая что $ U(s(x), y) = V(x, y) $. По теореме о неподвижной точке, найдется такое $ n $, что $ s(n) = n $. Отсюда, $ U(n, y) = U(s(n), y) = V(n, y) = ny $. \\ \\ \\








%%%----------%%%----------%%%----------%%%----------%%%



\textbf{\large Задача 3}
\medskip\hrule\medskip
\textit{Пусть $ U(p, x) - $ главная универсальная вычислимая функция, а $ V(n, x) - $  вычислимая функция от двух аргументов. Докажите, что найдется такое $ p $, что $ U(p, x) = V(p, x) $ для всех $ x $.} \\ \\

Так как $ U - $ ГУВФ, то существует функция $ s(x) $, такая что  $ \Rightarrow U(s(p), x) = V(p, x)$. По теореме о неподвижной точки, найдется $ p \; : s(p) = p \Rightarrow U(p, x) = U(s(p), x) = V(p, x) $.
\newpage




%%%----------%%%----------%%%----------%%%----------%%%



\textbf{\large Задача 4}
\medskip\hrule\medskip
\textit{Существует ли такая главная универсальная вычислимая функция $ U(p, x),$ в которой множество программ I, вычисляющих определенные в 0 функции, совпадает с множеством четных чисел?} \\ \\ 

Будем решать от противного. Допустим это так, отсюда множество таких функций разрешимо. Так же можем воспользоваться теоремой Успенского - Райса, взяв за нетривиальное свойство как раз определенность в 0. Получаем, что оно неразрешимо. Противоречие.
\\ \\ \\




%%%----------%%%----------%%%----------%%%----------%%%



\textbf{\large Задача 6}
\medskip\hrule\medskip
\textit{Пусть $ U(p, x) - $ главная универсальная вычислимая функция. Обозначим через $ K \subset \mathbb{N}^2 $ множество таких пар$ (k, n) $, что функция $ U_k(x) = U(k, x) $ является продолжением функции $ U_n(x) = U(n, x) $, то есть $  U_k(x) = U_n(x), $ где $ U_n(x) - $ определена. Докажите, что $ K $ - неразрешимо.} \\ \\ 

Рассмотрим функции вида $ f_i(n) = n, \; f_i(i) - $ не определено. Построим $ K' $ как первые координаты $ K $. Свойство продолжения для $ n $, описанное в условии задачи, является нетривиальным, отсюда, пользуясь теоремой Успенского - Райса, $ K' - $ неразрешимо. Теперь предположим, что $ K - $ разрешимое. Всего пар $ - \mathbb{N}^2 $ счетно много. Тогда будем проверять лежит ли каждая пара в $ K $, и если лежит, то выписывать ее первую координату. Таким образом, окажется что $ K' -  $ перечислимо. Противоречие.  


















\end{document}