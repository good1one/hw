\documentclass[a4paper,11pt]{article}
\usepackage[T1]{fontenc}
\usepackage[utf8]{inputenc}
\usepackage{graphicx}
\usepackage{xcolor}


\usepackage{amsmath,amssymb,amsthm,textcomp}
\usepackage{enumerate}
\usepackage{multicol}
\usepackage{tikz}
\usepackage[english, russian]{babel}

\usepackage{geometry}
\geometry{total={210mm,297mm},
	left=25mm,right=25mm,%
	bindingoffset=0mm, top=20mm,bottom=20mm}


% custom footers and headers
\usepackage{fancyhdr}
\pagestyle{fancy}
\lhead{}
\chead{}
\rhead{}
\lfoot{}
\cfoot{}
\rfoot{\fbox{\thepage}}
\renewcommand{\headrulewidth}{0pt}
\renewcommand{\footrulewidth}{0pt}

%%%----------%%%----------%%%----------%%%----------%%%

\begin{document}
\textbf{\large Задача 1}
\medskip\hrule\medskip
Для нашего удобства перейдем к алфавиту $ \{0, 1, 2\} $. \\[2pt]
Тогда от нашего алгоритма требуется ни что иное, как прибавлять в числу, записанному в троичной системе, 1, если оно не целиком состоит из двоек, и выводить $ size + 1 $ ноль в обратном случае. В целом, это классический алгоритм, поэтому мы опишем его кратко.  \\[3pt]
В состояние $ q_0 $ двигаемся в право до пустого символа и переходим в состояние $ q_1 $. В этом состояние движемся влево, пропуская все двойки и заменяя их на 0. Далее, если мы встречаем 1 или 0, делаем из них $ n + 1 $, переходим в новое состояние, упора в право и в финальное. Если же таких не встретилось, то заменяем пустой символ на 0, и снова переходим в финальное.
\\ \\ \\



%%%----------%%%----------%%%----------%%%----------%%%


\textbf{\large Задача 2}
\medskip\hrule\medskip
Запустим нашу машину на $ |Q \times A| $ шага. Если головка сдвинется, ответ отрицательный. Если не сдвинется, то какая-то пара вида $ (q,a) $ за это время повторится, тогда машина войдёт в цикл, и будет продолжать делать то же самое. В этом случае ответ положительный, поэтому ответ да.
\\ \\ \\



%%%----------%%%----------%%%----------%%%----------%%%


\textbf{\large Задача 3}
\medskip\hrule\medskip
Запустим нашу машину на $ |Q \times A| $ шага. Если  машина за это время остановилась,  мы даем ответ . Если не остановилась, то она дважды побывала в одном и том же состоянии. Это значит, что она и далее через то же число шагов в него придет независимо от того, сместилась ли головка. В этом случае машина будет работать бесконечно долго, даем ответ нет. Отсюда, множество таких МТ разрешимо.



\end{document}