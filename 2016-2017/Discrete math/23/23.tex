\documentclass[a4paper,11pt]{article}
\usepackage[T1]{fontenc}
\usepackage[utf8]{inputenc}
\usepackage{graphicx}
\usepackage{xcolor}


\usepackage{amsmath,amssymb,amsthm,textcomp}
\usepackage{enumerate}
\usepackage{multicol}
\usepackage{tikz}
\usepackage[english, russian]{babel}

\usepackage{geometry}
\geometry{total={210mm,297mm},
	left=25mm,right=25mm,%
	bindingoffset=0mm, top=20mm,bottom=20mm}


% custom footers and headers
\usepackage{fancyhdr}
\pagestyle{fancy}
\lhead{}
\chead{}
\rhead{}
\lfoot{}
\cfoot{}
\rfoot{\fbox{\thepage}}
\renewcommand{\headrulewidth}{0pt}
\renewcommand{\footrulewidth}{0pt}

%%%----------%%%----------%%%----------%%%----------%%%

\begin{document}
\textbf{\large Задача 1}
\medskip\hrule\medskip
\textit{Постройте МТ, которая вычисляет нигде не определенную функцию.} \\ \\

Заметим, что если функция нигде не определена, то, это значит,  что она никогда не завершает свою работу.  Определим алфавит нашей МТ за $ \{0, \Lambda \} $.
\begin{align*}
	&(0, q_0) \rightarrow (0, q_0, 0) \\
	&(\Lambda \quad q_0) \rightarrow (\Lambda \quad q_0 \quad 0) \\
\end{align*}
Алгоритм корректен, так как на любом входе машина циклится, а это и есть то, что нам нужно.
\\ \\ \\



%%%----------%%%----------%%%----------%%%----------%%%




\textbf{\large Задача 2}
\medskip\hrule\medskip
\textit{Постойте МТ, которая инвертирует входное двоичное число:  на входе $ w, $ где $ w = w_1 \dots w_n, \; w_i \in \{0, 1\}  $, результатом работы должно быть слово $ \overline{w} = \overline{w_1} \dots \overline{w_n}. $} \\

\begin{align*}
	&(0, q_1) \rightarrow (1, q_1, +1) \\
	&(1, q_1) \rightarrow (0, q_1, +1) \\
	&(\Lambda, q_1) \rightarrow (\Lambda, q_2, -1) \\[10pt]
	&(0, q_2) \rightarrow (0, q_2, -1) \\
	&(1, q_2) \rightarrow (1, q_2, -1) \\
	&(\Lambda, q_2) \rightarrow (\Lambda, q_f, 0 ) 
\end{align*}
Алгоритм корректен, так как, действительно, мы просто бежим по числу и инвертируем последовательно каждый из его разрядов, после чего, возвращаемся обратно.
\newpage





%%%----------%%%----------%%%----------%%%----------%%%




\textbf{\large Задача 3}
\medskip\hrule\medskip
\textit{Постройте МТ, проверяющую, входит ли слово в алфавите $ \{a, b, c\} $ подслово $ aba $. В конце работы машины на ленте должно остаться 1, если такое подслово есть и 0, если его нет.} \\


\begin{align*}
	&(b, q_1) \rightarrow (\Lambda, q_1, +1) \\
	&(c, q_1) \rightarrow (\Lambda, q_1, +1) \\
	&(a, q_1) \rightarrow (\Lambda, q_2, +1) \\[10pt]
	&(a, q_2) \rightarrow (\Lambda, q_2, +1) \\
	&(c, q_2) \rightarrow (\Lambda, q_1, +1) \\
	&(b, q_2) \rightarrow (\Lambda, q_3, +1) \\[10pt]
	&(b, q_3) \rightarrow (\Lambda, q_1, +1) \\
	&(c, q_3) \rightarrow (\Lambda, q_1, +1) \\
	&(a, q_3) \rightarrow (1, q_4, +1) \\[10pt]
	&(a, q_4) \rightarrow (\Lambda, q_4, +1) \\	
	&(b, q_4) \rightarrow (\Lambda, q_4, +1) \\
	&(c, q_4) \rightarrow (\Lambda, q_4, +1) \\[10pt]
	&(\Lambda, q_1) \rightarrow (0, q_f, 0) \\
	&(\Lambda, q_2) \rightarrow (0, q_f, 0) \\
	&(\Lambda, q_3) \rightarrow (0, q_f, 0) \\
	&(\Lambda, q_4) \rightarrow (1, q_f, 0) \\
\end{align*}
Докажем корректность нашего алгоритма. Для построения такой функции, нам нужно идти по слову, и если мы встречаем символ $ a $, то потенциально мы можем встретить наше подслово. Перейдем в новое состояние. Проверим второй символ. Аналогично проверим третий. Если символ не совпадает, то опять перейдем в начальное состояние. Если мы дошли до конца, значит подслово не встретилось - выводим 0. И наоборот, если оно встречалось, то из состояния, в котором уже встретилось все подслово, выводим 1. 
\newpage


%%%----------%%%----------%%%----------%%%----------%%%




\textbf{\large Задача 4}
\medskip\hrule\medskip
\textit{Докажите, что сущесвует МТ, которая сортирует символы входного двоичного числа: на входе $ w $, где двоичное слово $ w $  содержит $ a $ нулей и $ b $ единиц, результатом работы должно быть слово $ 0^{a}1^{b} $.} \\ \\

Представим неявно наш алгоритм. Будем просто идти по элементам. В случае, когда мы встретили 1, заменим ее на $ \Lambda  $. Перейдем в новое состояние, и пойдем снова по элементам, выполняя поиск уже 0. Заменяем его на 1. Далее перейдем в новое состояние, вернемся назад до $ \Lambda $ и заменим ее на 0. И так далее. Если встеретили символ конца ввода, возвращаемся назад до первого символа и переходим в финальное состояние.  \\[2pt]
Корректность алгоритма почти очевидна, так как на каждом шаге алгоритма мы избавляемся хотя бы от 1 инверсии.  \\ \\ \\



%%%----------%%%----------%%%----------%%%----------%%%




\textbf{\large Задача 5}
\medskip\hrule\medskip
\textit{Докажите существование МТ, которая проверяет, что вход является полиндромом. Если является , результат работы должен быть 1, а если нет, то результат 0.} \\ \\

Для удобства будем использовать 2 ленты. Скопируем входное слово на 2 ленту и переместим головку на последнюю позицию.  \\[2pt]
Определим наши переходы. Функция от двух равных аргументов переводит головку первой ленты направо, второй - налево. При этом символ на первой ленте переходит в пустой, символ на второй - сам в себя. \\[2pt] 
Если же аргументы не равны, перейдем в новое состояние, при этом  с символами мы будем работать аналогично. \\[2pt]
Дойдя до пустого символа, проверим находимся ли мы в начальном состояние, если да, то выводим 1 на перувую ленту, нет - 0. Таким образом, мы неявно предъявили такой алгоритм.
\\ \\ \\




%%%----------%%%----------%%%----------%%%----------%%%




\textbf{\large Задача 6}
\medskip\hrule\medskip
\textit{Существует ли машинв Тьюринга, при начале работы на пустой ленте, оставляющая на ней 2017 единиц и имеющая не более 100 состояний?} \\ \\

Для нашего удобства заведем 2018 лент и запишем в последние 2017 0. Первую ленту будем использовать для выписания ответа. Определим наши переходы как мы принимаем на вход значения от 2017 лент, на $ i $-ом шаге которых первые $ i - 1 $ значений от  вспомогательных функций равняются $ \Lambda $. Оно перейдет в аналогичную штуку с $ i $ лямбдами. При этом каретка первой ленты выписывает 1 и смещается на одну позицию вправо, так же как и $ i $ ая каретка $ i $ из 2017 ленты. Заканчиваем алгоритм, когда на вход получаем 2017 лямбд, и возвращаем каретку первой ленты в начало. Таким образом, мы как раз выпишем 2017 единиц.
\\ \\ \\





%%%----------%%%----------%%%----------%%%----------%%%




\textbf{\large Задача 7}
\medskip\hrule\medskip
\textit{Докажите существование машины Тьюринга, вычисляющую какую-либо биекцию между $ \mathbb{N} \times \mathbb{N} $ и $ \mathbb{N} $. Т. е. в начале $ 1^a\#1^b $, а в конце $ 1^{f(a, b)} $, где $ f $ - выбранная вами биекция.} \\ \\

$ f(s) = 2^i(2j + 1) $ - биекция. Пользуясь тезисом Черча - Тьюринга, получаем, что данная биекция вычислима на МТ.







\end{document}