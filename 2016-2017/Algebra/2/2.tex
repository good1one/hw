\documentclass[a4paper,11pt]{article}
\usepackage[T1]{fontenc}
\usepackage[utf8]{inputenc}
\usepackage{graphicx}
\usepackage{xcolor}


\usepackage{amsmath,amssymb,amsthm,textcomp}
\usepackage{enumerate}
\usepackage{multicol}
\usepackage{tikz}
\usepackage[english, russian]{babel}

\usepackage{geometry}
\geometry{total={210mm,297mm},
	left=25mm,right=25mm,%
	bindingoffset=0mm, top=20mm,bottom=20mm}


% custom footers and headers
\usepackage{fancyhdr}
\pagestyle{fancy}
\lhead{}
\chead{}
\rhead{Егерев Артем}
\cfoot{}
\renewcommand{\headrulewidth}{0pt}
\renewcommand{\footrulewidth}{0pt}

%%%----------%%%----------%%%----------%%%----------%%%

\begin{document}
\textbf{\large Задача 1}
\medskip\hrule\medskip
\textit{Найдите все левые смежные классы и все правые смежные классы группы $ A_4 $ по подгруппе $ H = \;  \langle\sigma\rangle $, где $ \sigma = \begin{pmatrix}
	1 & 2 & 3 & 4 \\[2pt]
	2 & 1 & 4 & 3
	\end{pmatrix} $. Является ли подгруппа $ H $ нормальной в группе $ A_4 $? } \\ \\ 
Для начала определим нашу подгруппу $ H $:
\begin{align*}
&b_1 = 
\begin{pmatrix}
1 & 2 & 3 & 4 \\[2pt]
1 & 2 & 3 & 4
\end{pmatrix}  = id \\
&b_2 = 
\begin{pmatrix}
1 & 2 & 3 & 4 \\[2pt]
2 & 1 & 4 & 3
\end{pmatrix} \\
&b_3 = 
\begin{pmatrix}
1 & 2 & 3 & 4 \\[2pt]
1 & 2 & 3 & 4
\end{pmatrix} = b_1
\end{align*}
Отсюда, группа $ H $ состоит всего из двух элементов. Рассмотрим группу $ A_4 $:
\begin{align*}
&a_{1} = ( 1, 2, 3, 4 ) \quad 
a_{2} = ( 1, 3, 4, 2 ) \quad 
a_{3} = ( 1, 4, 2, 3 ) \quad  \\
&a_{4} = ( 2, 1, 4, 3 ) \quad 
a_{5} = ( 2, 3, 1, 4 ) \quad 
a_{6} = ( 2, 4, 3, 1 ) \quad  \\
&a_{7} = ( 3, 1, 2, 4 ) \quad 
a_{8} = ( 3, 2, 4, 1 ) \quad 
a_{9} = ( 3, 4, 1, 2 ) \quad  \\
&a_{10} = ( 4, 1, 3, 2 ) \quad 
a_{11} = ( 4, 2, 1, 3 ) \quad 
a_{12} = ( 4, 3, 2, 1 ) \quad
\end{align*}
Левый класс  для элемента $ b_1 $ тривиальна и равна просто $ A $. Левый класс относительно элемента $ b_2 $ : $ \{ c_i \; : \; c_i = b_2 \circ a_i \}  $
\begin{align*}
&c_{1} = ( 2, 1, 4, 3 ) \quad 
c_{2} = ( 2, 4, 3, 1 ) \quad 
c_{3} = ( 2, 3, 1, 4 ) \quad  \\
&c_{4} = ( 1, 2, 3, 4 ) \quad 
c_{5} = ( 1, 4, 2, 3 ) \quad 
c_{6} = ( 1, 3, 4, 2 ) \quad  \\
&c_{7} = ( 4, 2, 1, 3 ) \quad 
c_{8} = ( 4, 1, 3, 2 ) \quad 
c_{9} = ( 4, 3, 2, 1 ) \quad  \\
&c_{10} = ( 3, 2, 4, 1 ) \quad 
c_{11} = ( 3, 1, 2, 4 ) \quad 
c_{12} = ( 3, 4, 1, 2 ) \quad
\end{align*} 
Аналогично правый класс для элемента $ b_1 $ тривиален. Для элемента $ b_2 $ : $ \{ d_i \; : \; d_i = a_i \circ b_2 \}  $
\begin{align*}
&d_{1} = ( 2, 1, 4, 3 ) \quad 
d_{2} = ( 3, 1, 2, 4 ) \quad 
d_{3} = ( 4, 1, 3, 2 ) \quad  \\
&d_{4} = ( 1, 2, 3, 4 ) \quad 
d_{5} = ( 3, 2, 4, 1 ) \quad 
d_{6} = ( 4, 2, 1, 3 ) \quad  \\
&d_{7} = ( 1, 3, 4, 2 ) \quad 
d_{8} = ( 2, 3, 1, 4 ) \quad 
d_{9} = ( 4, 3, 2, 1 ) \quad  \\
&d_{10} = ( 1, 4, 2, 3 ) \quad 
d_{11} = ( 2, 4, 3, 1 ) \quad 
d_{12} = ( 3, 4, 1, 2 ) \quad
\end{align*}
Отсметим, что для элемента, к примеру 7, $ gH \neq Hg $. А это, из утверждения, доказанного на лекции, дает нам, что подгруппа $ H $  не является нормальной.
\newpage



%%%----------%%%----------%%%----------%%%----------%%%


\textbf{\large Задача 2}
\medskip\hrule\medskip
\textit{Пусть $ SL_2 $ группа всех целочисленный $ (2 \times 2) $ матриц с определителем 1. Докажите, что множество}
\begin{gather*}
H = \{ \begin{pmatrix}
a & b \\
c & d
\end{pmatrix} \in 
SL_2(\mathbb{Z}) | a \equiv d \equiv 1 \; (mod \; 3) \; and \; b \equiv c \equiv 0 \; (mod \; 3) \}
\end{gather*}
является нормальной подгруппой в $ SL_2(\mathbb{Z}) $. \\ \\
Сначала докажем, что это подгруппа, для этого нужно проверить лежит ли произведение в этой подгруппе, лежит ли там обратный, а так же нейтральный элемент:\\
\begin{gather*}
\begin{pmatrix}
c_1 & c_2 \\
c_3 & c_4
\end{pmatrix}
= 
\begin{pmatrix}
a_1 & a_2 \\
a_3 & a_4
\end{pmatrix}
\begin{pmatrix}
b_1 & b_2 \\
b_3 & b_4
\end{pmatrix}
\end{gather*}
Тогда:
\begin{align*}
&c_1 = a_1 b_1 + a_2 b_3 \equiv a_1b_1 \equiv 1 \; mod \; 3 \\
&c_2 = a_1 b_2 + a_2 b_4 \equiv 0 + 0 \equiv 0 \; mod \; 3 \\
&c_3 = a_3 b_1 + a_4 b_3 \equiv a_3b_1 \equiv 1 \; mod \; 3 \\ 
&c_4 = a_3 b_2 + a_4 b_4 \equiv 0 + 0 \equiv 0 \; mod \; 3 \\
\end{align*}
\begin{gather*}
\begin{pmatrix}
c_1 & c_2 \\
c_3 & c_4
\end{pmatrix}^{-1} = 
\begin{pmatrix}
c_4 & -c_2 \\
-c_3 & c_1
\end{pmatrix}
\end{gather*}
\begin{align*}
&c_1 \equiv c_4 \equiv 1 \; mod \; 3 \\
&c_2 \equiv c_3 \equiv -c_3 \equiv -c_3  \equiv 0 \; mod \; 3 \\
\end{align*}
\begin{gather*}
\begin{pmatrix}
1 & 0 \\
0 & 1
\end{pmatrix}
\quad 
1 \equiv 1 \; mod \; 3 
\quad 0 \equiv 0 \; mod \; 3
\end{gather*}
Доказали. При этом обратная матрца ищется при помощи формулы Крамера. Как было показано на лекции, нам достаточно доказать, что $ shs^{-1} \in H $ для всех $ s \in SL_2$ и $ h \in H$. 
\begin{gather*}
s = \begin{pmatrix}
a_1 & a_2 \\
a_3 & a_4
\end{pmatrix}
\quad
h =  \begin{pmatrix}
b_1 & b_2 \\
b_3 & b_4
\end{pmatrix} \quad
s^{-1} = \begin{pmatrix}
a_4 & -a_2 \\
-a_3 & a_1
\end{pmatrix} \quad 
\end{gather*}
При этом $ a_1 a_4 - a_2 a_3 = 1, \; b_1 \equiv b_4 \equiv 1 \; and \; b_2 \equiv b_3 \equiv 0 \; mod \; 3 $. Пусть 
\begin{gather*}
shs^{-1} = \begin{pmatrix}
d_1 & d_2 \\ 
d_3 & d_4
\end{pmatrix}
\end{gather*}
Тогда:
\begin{align*}
&d_1 = (a_1b_1 + a_2b_3)a_4 - (a_1b_2 + a_2b_4)a_3 \equiv a_1a_4b_1 - a_2a_3b_4 \equiv 1 \; mod \; 3 \\
&d_2 = -(a_1b_1 + a_2b_3)a_2 + (a_1b_2 + a_2b_4)a_1 \equiv -a_1a_2b_1 + a_1a_2b_4 \equiv 0 \; mod \; 3 \\
&d_3 = (a_3b_1 + a_4b_3)a_4 - (a_3b_2 + a_4b_4)a_3 \equiv a_3a_4b_1 - a_3a_4b_4 \equiv 0 \; mod \; 3 \\
&d_4 = -(a_3b_1 + a_4b_3)a_2 + (a_3b_2 + a_4b_4)a_1 \equiv -a_3a_2b_1 + a_1a_4b_4 \equiv 1 \; mod \; 3 
\end{align*}
$ \Rightarrow shs^{-1} \in H $ \newpage



%%%----------%%%----------%%%----------%%%----------%%%


\textbf{\large Задача 3}
\medskip\hrule\medskip
\textit{Найдите все гомоморфизмы из группы $ \mathbb{Z}_{12} $ в группу $ \mathbb{Z}_{16} $.} \\ \\ 
Пусть $ \varphi : \mathbb{Z}_{12} \to \mathbb{Z}_{16}$. Обозначим $ \varphi(1) $ за  $ \gamma $. Тогда: 
\begin{gather*}
	0 = \varphi(0) = \varphi(12) = 12\gamma \Rightarrow 12\gamma \equiv 0 \; mod \;16 
\end{gather*}
Нужно заметить, что наше отображение задается единственным числом $ \gamma $, так как имея его вычисляем: $ \varphi(x) = x \varphi(1) = x\gamma $. Из формулы выше, следует, что $ \gamma \; \vdots \; 4$, а таких, различных по модулю 16 : $ \{0, 4, 8, 12\} $, то есть всего 4 штуки. Так же нам нужно проверить, что данная операция действительно является гомоморфизмом, то есть:
\begin{gather*}
x \equiv x' \; mod \; 12 \; \Rightarrow \varphi(x) \equiv \varphi (x') \; mod \; 16 \; \\
x \equiv x' \; mod \; 12 \Rightarrow x - x' \; \vdots \; 12 \quad and \quad \varphi(1) \; \vdots \; 4 \\
\varphi(x) - \varphi(x') = (x - x')\varphi(1) \equiv 0 \; mod \; 16 \;  
\end{gather*} \\ \\ \\



%%%----------%%%----------%%%----------%%%----------%%%


\textbf{\large Задача 4}
\medskip\hrule\medskip
\textit{Перечислите все, с точностью до изоморфизма группы, каждая из которых изоморфна любой своей неединичной подгруппе. } \\ \\
Рассмотрим несколько вариантов. $ G $ бесконечна. Возьмем произвольный элемент $ g \in G $. Тогда $ H = \langle g \rangle $ будет являться бесконечной циклической группой и будет равномощно $ \mathbb{Z} $. Отсюда $ G \simeq \mathbb{Z} $.  Так же нужно отметить, что все подгруппы в $ \mathbb{Z} $ имеют вид $ \Rightarrow k\mathbb{Z} \simeq \mathbb{Z}$.\\[2pt]
$ G $ конечна. $ G \simeq \mathbb{Z}_n $. Заметим, что в данном случае  любая подгруппа будет иметь мощность меньше чем $ G $, или совпадать с ней, причем совпадение бывает в том и только в том случае, когда $ n \; -$ есть простое число. Очевидно, что когда множества совпадают, то они изоморфны.












\end{document}