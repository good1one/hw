\documentclass[a4paper,11pt]{article}
\usepackage[T1]{fontenc}
\usepackage[utf8]{inputenc}
\usepackage{graphicx}
\usepackage{xcolor}


\usepackage{amsmath,amssymb,amsthm,textcomp}
\usepackage{enumerate}
\usepackage{multicol}
\usepackage{tikz}
\usepackage[english, russian]{babel}

\usepackage{geometry}
\geometry{total={210mm,297mm},
	left=25mm,right=25mm,%
	bindingoffset=0mm, top=20mm,bottom=20mm}


% custom footers and headers
\usepackage{fancyhdr}
\pagestyle{fancy}
\lhead{}
\chead{}
\cfoot{}
\newcommand{\F}{\mathbb{F}}
\newcommand{\Z}{\mathbb{Z}}
\renewcommand{\headrulewidth}{0pt}
\renewcommand{\footrulewidth}{0pt}




%%%----------%%%----------%%%----------%%%----------%%%




\begin{document}
\textbf{\large Задача 1}
\medskip\hrule\medskip
\textit{Постройте явное поле $ \F_8 $ и составьте для него таблицы сложения и умножения.} \\ \\
Отметим тот факт, что поле $ \F_8 \simeq \F_2[x] / x^3 \simeq $ остаткам при деление на $ x^3 + x^2 + 1 $, то есть многочлены степени не более чем 2 над полем 2. По данному отображению легко строим таблицу сложения и умножения:
\begin{gather*}
	\begin{array}{|c|c|c|c|c|c|c|c|c|}
	\hline
	+			& 0 & 1 & x & x + 1 & x^2 & x^2 + x & x^2 + 1 &  x^2 + x + 1\\
	\hline
	0			& 0 & 1 & x & x + 1 & x^2 & x^2 + x & x^2 + 1 &  x^2 + x + 1\\
	\hline
	1			& 1 & 0 & x + 1 & x & x^2 + 1 & x^2 + x + 1 & x^2 &  x^2 + x\\
	\hline
	x			& x & x + 1 & 0 & 1 & x^2 + x & x^2 & x^2 + x + 1 &  x^2 + 1\\
	\hline
	x + 1		& x + 1 & x & 1 & 0 & x^2 + x + 1 & x^2 + 1 & x^2 + x &  x^2\\
	\hline
	x^2			& x^2 & x^2 + 1 & x^2 + x & x^2 + x + 1 & 0 & x & 1 & x + 1\\
	\hline
	x^2 + x		& x^2 + x & x^2 + x + 1 & x^2 & x^2 + 1 & x & 0 & x + 1 & 1\\
	\hline
	x^2 + 1		& x^2 + 1 & x^2 & x^2 + x + 1 & x^2 + x & 1 & x + 1 & 0 & x\\
	\hline
	x^2 + x + 1	& x^2 + x + 1 & x^2 + x & x^2 + 1 & x^2 & x + 1 & 1 & x & 0\\
	\hline
\end{array}
\end{gather*}
\begin{gather*}
	\begin{array}{|c|c|c|c|c|c|c|c|c|}
	\hline
	\times		& 0 & 1 & x & x + 1 & x^2 & x^2 + x & x^2 + 1 &  x^2 + x + 1\\
	\hline
	0			& 0 & 0 & 0 & 0 & 0 & 0 & 0 &  0\\
	\hline
	1			& 0 & 1 & x & x + 1 & x^2 & x^2 + x & x^2 + 1 &  x^2 + x + 1\\
	\hline
	x			& 0 & x & x^2 & x^2 + x & x^2 + 1 & 1 & x^2 &  x + 1\\
	\hline
	x + 1		& 0 & x + 1 & x^2 + x & x^2 + 1 & x + 1 & x^2 + x + 1 & x &  x^2\\
	\hline
	x^2			& 0 & x^2 & x^2 + 1 & 1 & x^2 + x + 1 & x & x + 1 &  x^2 + x\\
	\hline
	x^2 + x		& 0 & x^2 + x & 1 & x^2+x+1 & x & x + 1 & x^2 &  x^2 + 1\\
	\hline
	x^2 + 1		& 0 & x^2 + 1 & x^2 + x + 1 & x & x + 1 & x^2 & x^2 + x & 1\\
	\hline
	x^2 + x + 1	& 0 & x^2 + x + 1 & x + 1 & x^2 & x^2 + x & x^2 + 1 & 1 & x\\
	\hline
	\end{array}
\end{gather*} \\ \\ \\



\textbf{\large Задача 2}
\medskip\hrule\medskip
\textit{Реализуем поле $ \F_9 $ в виде $ \Z_3[x]/(x^2 + 1) $. Перечислите в этой реализации все элементы данного поля, являющиеся пораждающими циклической группы $ \F_9^{\times} $.  } \\ \\
Для начала построим таблицу умножения:
\begin{gather*}
	\begin{array}{|c|c|c|c|c|c|c|c|c|c|}
	\hline
	\times	& 0 & 1 & 2 & x & x + 1 & x + 2 & 2x & 2x + 1 & 2x + 2\\
	\hline
	0		& 0 & 0 & 0 & 0 & 0 & 0 & 0 & 0 & 0\\
	\hline
	1		& 0 & 1 & 2 & x & x + 1 & x + 2 & 2x & 2x + 1 & 2x + 2\\
	\hline
	2		& 0 & 2 & 1 & 2x & 2x + 2 & 2x + 1 & x & x + 2 & x + 1\\
	\hline
	x		& 0 & x & 2x & 2 & x + 2 & 2x + 2 & 1 & x + 1 & 2x + 1\\
	\hline
	x + 1	& 0 & x + 1 & 2x + 2 & x + 2 & 2x & 1 & 2x + 1 & 2 & x\\
	\hline
	x + 2	& 0 & x + 2 & 2x + 1 & 2x + 2 & 1 & x & x + 1 & 2x & 2\\
	\hline
	2x		& 0 & 2x & x & 1 & 2x + 1 & 1 + x & 2 & 2x + 2 & x + 2\\
	\hline
	2x + 1	& 0 & 2x + 1 & x + 2 & x + 1 & 2 & 2x & 2x + 2 & x & 1\\
	\hline
	2x + 2	& 0 & 2x + 2 & x + 1 & 2x + 1 & x & 2 & x + 2 & 1 & 2x\\
	\hline
	\end{array}
\end{gather*}
Отметим, что порядок пораждающих элементов равен 8, откуда получаем ответ: \\ $x + 1, x + 2, 2x + 1, 2x + 2$ 
\\ \\ \\




\textbf{\large Задача 3}
\medskip\hrule\medskip
\textit{Проверьте, что многочлен $ x^2 + 1 $ и $ y^2 - y - 1 $ непреводимы над $ \Z_3 $, и установите явно изоморфизм между $ \Z_3[x]/(x^2 + 1) $ и $ \Z_3[y]/(y^2 - y - 1) $.} \\ \\
$ x^2 + 1 = \{1, 2\} $ над $ \Z_3 \Rightarrow $ корней нет. $ y^2 - y - 1 = y^2 + 2y + 2 = (y + 1)^2 + 1 = \{1, 2\}$, так же без корней. \\[2pt]
Построим таблицу умножения в $ \Z_3[y]/(y^2 - y - 1) $:
\begin{gather*}
	\begin{array}{|c|c|c|c|c|c|c|c|c|c|}
	\hline
	\times	& 0 & 1 & 2 & y & y + 1 & y + 2 & 2y & 2y + 1 & 2y + 2\\
	\hline
	0		& 0 & 0 & 0 & 0 & 0 & 0 & 0 & 0 & 0\\
	\hline
	1		& 0 & 1 & 2 & y & y + 1 & y + 2 & 2y & 2y + 1 & 2y + 2\\
	\hline
	2		& 0 & 2 & 1 & 2y & 2y + 2 & 2y + 1 & y & y + 2 & y + 1\\
	\hline
	y		& 0 & y & 2y & y + 1 & 2y + 1 & 1 & 2y+2 & 2 & y + 2\\
	\hline
	y + 1	& 0 & y + 1 & 2y + 2 & 2y + 1 & 2 & y & y + 2 & 2y & 1\\
	\hline
	y + 2	& 0 & y + 2 & 2y + 1 & 1 & y & 2y+2 & 2 & y + 1 & 2y\\
	\hline
	2y		& 0 & 2y & y & 2y+2 & y +2  & 2 & y+1 & 1 & 2y+1\\
	\hline
	2y + 1	& 0 & 2y + 1 & y + 2 & 2 & 2y & y+1 & 1 & 2y+2 & y\\
	\hline
	2y + 2	& 0 & 2y + 2 & y + 1 & y+2 & 1 & 2y & 2x + 1 & y & 2\\
	\hline
	\end{array}
\end{gather*}
Для корректного определения изоморфизма, нам достаточно указать,  куда переходит $ 1 $ и $ x $. Учитываю порядки элементов получаем:
\begin{gather*}
\varphi(1) = 1 \qquad \varphi(x) = y + 1 
\end{gather*}






\end{document}