\documentclass[a4paper,11pt]{article}
\usepackage[T1]{fontenc}
\usepackage[utf8]{inputenc}
\usepackage{graphicx}
\usepackage{xcolor}


\usepackage{amsmath,amssymb,amsthm,textcomp}
\usepackage{enumerate}
\usepackage{multicol}
\usepackage{tikz}
\usepackage[english, russian]{babel}

\usepackage{geometry}
\geometry{total={210mm,297mm},
	left=25mm,right=25mm,%
	bindingoffset=0mm, top=20mm,bottom=20mm}


% custom footers and headers
\usepackage{fancyhdr}
\pagestyle{fancy}
\lhead{}
\chead{}
\cfoot{}
\renewcommand{\headrulewidth}{0pt}
\renewcommand{\footrulewidth}{0pt}




%%%----------%%%----------%%%----------%%%----------%%%




\begin{document}
\textbf{\large Задача 1}
\medskip\hrule\medskip
\textit{Найти НОД многочленов}
\begin{align*}
f(x) = 2x^4  - 4x^3 - 3x^2 + 7x - 2 \qquad
g(x) = 6x^3 + 4x^2 - 5x + 1
\end{align*}
\textit{а так же его линейное выражение через $ f(x) $ и $ g(x) $.} \\ \\
Для нахождения НОДа воспользуемся теоремой Евклида:
\begin{enumerate}
	\item $ f(x) = (\frac13 x - \frac89 )g(x) + (\frac{20}9 x^2 + \frac{20}9x - \frac{10}9 ) $
	\item $ g(x) = (\frac{27}{10}x - \frac9{10})(\frac{20}9 x^2 + \frac{20}9x - \frac{10}9 ) $
\end{enumerate}
Как делятся многлены достаточно очевидно. Получаем, НОД = $ \frac{20}9 x^2 + \frac{20}9x - \frac{10}9  $.
\begin{gather*}
	\text{НОД } =  f(x) - (\dfrac13 x - \dfrac89 )g(x)
\end{gather*}
\\ \\ \\





%%%----------%%%----------%%%----------%%%----------%%%



\textbf{\large Задача 2}
\medskip\hrule\medskip
\textit{Разложите многочлен $ x^6 + x^3 - 12 $ в произведение неприводимых в кольце $ \mathbb{C}[x] $ и в кольце $ \mathbb{R}[x] $.} \\ \\
1. Начнем с $ \mathbb{R} $.
\begin{gather*}
	x^6 + x^3 - 12 = (x^3 + 4)(x^3 - 3) = (x + \sqrt[3]{4})(x^2 - \sqrt[3]{4}x + \sqrt[3]{16})(x - \sqrt[3]{3})(x^2 + \sqrt[3]{3}x + \sqrt[3]{9}) 
\end{gather*}
Причем очевидно, что дальше не раскладывается, так как $ D < 0 $. \\[3pt]
2. Теперь к $ \mathbb{C} $. Здесь неприводимыми уже будут многочлены степени не больше 1, поэтому наши квадраты разложатся еще на 2 слогаемых:
\begin{gather*}
		x^6 + x^3 - 12 = (x + \sqrt[3]{4})(x - \sqrt[3]{3})(x^2 - \sqrt[3]{4}x + \sqrt[3]{16})(x^2 + \sqrt[3]{3}x + \sqrt[3]{9}) = \\[2pt]
		(x + \sqrt[3]{4})(x - \sqrt[3]{3}) \cdot (x - \frac{1 - i\sqrt{3}}{\sqrt[3]{2}}) (x - \frac{1 + i\sqrt{3}}{\sqrt[3]{2}}) \cdot (x - \frac12 i \sqrt[3]{3}(i + \sqrt{3}))(x - \frac12 i \sqrt[3]{3}(i - \sqrt{3}))
\end{gather*}
Многочлен 6 степени - 6 корней, все отлично.
\\ \\ \\





%%%----------%%%----------%%%----------%%%----------%%%



\textbf{\large Задача 3}
\medskip\hrule\medskip
\textit{Выясните, является ли число $ 5 + \sqrt{-5} $ элементом кольца $ \mathbb{Z}[\sqrt{-5}]. $} \\ \\
$ Z[\sqrt{-5}] = \{ z : z = m + \sqrt{-5}n \quad m, \; n \in \mathbb{Z} \} $ \\
Рассмотрим два элемента и докажем их необратимость:
\begin{enumerate}
	\item $ 1 - \sqrt{-5} $ : \\
	Предположим, что существуют $ m, n \neq (0, 0) $, что $ (1 - \sqrt{-5})(m + \sqrt{-5}n) = 1 $, тогда
	\begin{gather*}		
	(1 - \sqrt{-5})(m + \sqrt{-5}n) = (m + 5n) + i\sqrt{5}(n - m) = 1 + 0i \Rightarrow n = m = \frac16  \notin \mathbb{Z} \Rightarrow \text{противоречие}
	\end{gather*}
	\item $ \sqrt{-5} $ : \\
	Аналогично.
	\begin{gather*}
		\sqrt{-5}( m + \sqrt{-5}n) = -5n + \sqrt{5}mi = 1 + 0i \Rightarrow m = 0, n = -\frac15 \notin \mathbb{Z} \Rightarrow \text{противоречие}
	\end{gather*}
\end{enumerate} 
Отметим, что $ 5 + \sqrt{-5} = \sqrt{-5}(1 - \sqrt{-5}) $, следовательно элемент раскладывается на 2 необратимых, следовательно не является простым. 
\newpage






%%%----------%%%----------%%%----------%%%----------%%%



\textbf{\large Задача 4}
\medskip\hrule\medskip
\textit{Пусть $ R $ - евклидово кольцо с нормой $ N $. Докажите, что $ N $ принимает бесконечное число значений.} \\ \\
Рассмотрим элемент $ m $, при котором норма достигает своего максимального значения (если такой существует). Дополнительно введем элемент $ n $ -  ненулевой необратимый. Такой найдется так как евкилидово кольцо это не поле. $ N(mn) > N(m) = max $, так как элемент $ n $ необратимый, теорема из лекции.
Получаем, что $ N $ не может быть ограничено, то есть принимает бесконечное число значений.
 









\end{document}