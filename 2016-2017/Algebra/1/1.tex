\documentclass[a4paper,11pt]{article}
\usepackage[T1]{fontenc}
\usepackage[utf8]{inputenc}
\usepackage{graphicx}
\usepackage{xcolor}


\usepackage{amsmath,amssymb,amsthm,textcomp}
\usepackage{enumerate}
\usepackage{multicol}
\usepackage{tikz}
\usepackage[english, russian]{babel}

\usepackage{geometry}
\geometry{total={210mm,297mm},
	left=25mm,right=25mm,%
	bindingoffset=0mm, top=20mm,bottom=20mm}


% custom footers and headers
\usepackage{fancyhdr}
\pagestyle{fancy}
\lhead{}
\chead{}
\rhead{Егерев Артем}
\lfoot{}
\cfoot{}
\renewcommand{\headrulewidth}{0pt}
\renewcommand{\footrulewidth}{0pt}

%%%----------%%%----------%%%----------%%%----------%%%

\begin{document}
\textbf{\large Задача 1}
\medskip\hrule\medskip
\textit{Докадите, что формула $ m \circ n = mn - m - n + 2 $ задает бинарную операцию на множестве $ \mathbb{Q} \setminus \{1\} $ и что $ (\mathbb{Q} \setminus \{1\}, \circ) $ является группой.} \\ \\
1) Сначала докажем, что $ m \circ n = mn - m - n + 2 $ является бинарной операцией.
\begin{align*}
	&m \circ n = mn - m - n + 2 = 
	m(n - 1) - n + 2 =
	m(n - 1) - n + 1 + 1 = 
	(m - 1)(n - 1) + 1 \\
	&m - 1 \in \mathbb{Q}, n - 1 \in \mathbb{Q} \Rightarrow (m - 1)(n - 1) \in \mathbb{Q} \Rightarrow	(m - 1)(n - 1) + 1 \in \mathbb{Q}
\end{align*}
Теперь остается показать, что 1 мы не получим. Действительно:
\begin{align*}
	1 = (m - 1)(n - 1) + 1 \Rightarrow m = 1 \; or \; n = 1 \Rightarrow \varnothing
\end{align*}
2) Для того, чтобы доказать, что $ \mathbb{Q} \setminus \{1\} $ группа необходимо проверить 3 свойства. 
\begin{gather*}
\text{a) ассоциативность :} \; 
(a \circ b) \circ c = (ab - a - b + 2) \circ c = abc - ac - bc + 2c - ab + a +  b - 2 + 2 =\\[2pt]
=  a(bc - b - c + 2) - a - bc + b + c - 2 + 2 = a \circ (b \circ c) \\[3pt]
\text{б) наличие нейтрального элемента : } \; \exists \; e = 2 : a \circ e = 2a - a - 2 + 2 = e \circ a = 2a = 2 - a + 2 = a  \\[3pt]
\text{в) наличие обратного элемента : } \; \exists \; a^{-1} = \frac{a}{a - 1} \neq 1  \; : a \circ a^{-1} = \frac{a^2}{a - 1} - \frac{a}{a - 1} - a + 2 =  
= a^{-1} \circ a = \\ = \frac{a^2}{a - 1} - a - \frac{a}{a - 1} + 2 = e = 2
\end{gather*}
\\ \\ \\ 

%%%----------%%%----------%%%----------%%%----------%%%


\textbf{\large Задача 2}
\medskip\hrule\medskip
\textit{Найдите порядки всех элементов группы $ \{\mathbb{Z}_{12}, +\} $}. \\ \\
Начнем с определения.  Порядком элемента $ g $ называется такое наименьшее число$\; m $, что $ g^m = e $. Если такого натурального $ m $ не сущетсвует, говорят, что порядок элемента $ g $ равен бесконечности . \\[2pt]
В нашем случае $ e = 0  \Rightarrow g^m \equiv gm \equiv 0 \; mod \; 12 \Rightarrow 12 \; | \;  gm \cup m = min \Rightarrow m = \frac{LCM(12, \; g)}{g}$ \\[2pt]
Далее для каждого элемента:
\begin{itemize}
\item 1 : LCM(12, 0) = 0, m = 1  
\item 2 : LCM(12, 1) = 12, m = 12 
\item 3 : LCM(12, 3) = 12, m = 4  
\item 4 : LCM(12, 4) = 12, m = 3 
\item 5 : LCM(12, 5) = 60, m = 12  
\item 6 : LCM(12, 6) = 12, m = 2 
\item 7 : LCM(12, 7) = 84, m = 12
\item 8 : LCM(12, 8) = 24, m = 3 
\item 9 : LCM(12, 9) = 36, m = 4 
\item 10 : LCM(12, 10) = 60, m = 5
\item 11 : LCM(12, 11) = 131, m = 12
\end{itemize}   
\newpage


%%%----------%%%----------%%%----------%%%----------%%%


\textbf{\large Задача 3}
\medskip\hrule\medskip
\textit{Опишите все подгруппы в группе $ \{\mathbb{Z}_{12}, +\} $}. \\ \\
Для начала, заметим, что если элемент $ a $ лежит в подгруппе, то и обратный к нему $ a^{-1} $ лежит в той же подгруппе. Так же во всех подгруппах обязан присутствовать нейтральный элемент $ e = 0 $. \\[2pt]  
По этому принципу все элементы можно разбить на пары обратных: $ 1 - 11, 2 - 10, 3 - 9, 4 - 8, \\ 5 - 7, 6 - 6$. \\[2pt] 
Тогда проверка на то, является ли данное множество подгруппой, сужается всего до проверки того факта, что каждая сумма двух элементов множества лежит в том же множетсве. \\[2pt]
Уже сходу можем назвать две подгруппы: это $ \{0\},  \{\mathbb{Z}_{12}, +\}$   \\[2pt]
Теперь, если в нашей подгруппе лежит 1, то там лежат и все элементы, так как мы можем сколь угодно раз складывать единицу саму с собой. Аналогично, если есть два числа, отличающиеся на 1, значит есть 1, а значит и все элементы. \\[2pt]
Все варианты с 1 мы уже рассмотрели, теперь если есть 2. Тогда есть $ 10, 4 - 8, 6 - 6 $. Это уже подгруппа (так все суммы не выходят из подгруппы). Ок. Могут ли быть еще элементы в такой подгруппе?
Нет. Если есть 3, получаем 2 и 3. Если 5, получаем 5 и 4. \\[2pt]
Перейдем к 3. Есть 3, а значит есть $ 9, 6 - 6 $. Тоже подгруппа, аналогично. Если добавим еще елементы: 4, то 3 и 4, 5 то 5 и 6. \\[2pt]
Далее 4. Добавляем сразу 8. Снова получилась группа. Если к этому добавить 5, то получится 4 и 5, если 6, то 4 + 6 = 10 $ \Rightarrow $ будет и 2, а этот случай уже рассмотрен. \\[2pt]
Если есть 5, то есть и 10, а значит и 2. Аналогично. \\[2pt]
6-ка нам подходит. \\[2pt]
Итого: 
\begin{gather*}
\{0\} \quad \{\mathbb{Z}_{12}\} \quad \{0, 2, 4, 6, 8, 10\} \quad \{0, 3, 6, 9\} \quad 
\{0, 4, 8\} \quad \{0, 6\}	
\end{gather*}  \\ \\ \\

%%%----------%%%----------%%%----------%%%----------%%%


\textbf{\large Задача 4}
\medskip\hrule\medskip
\textit{Докажите, что всякая бесконечная группа сожержит бесконечное число подгрупп} \\ \\
Заметим, что если мы возьмем $ g \; -$ элемент группы, $ g^{-1} \; -$ обратный к нему, то $ \{0, g, g^{-1}\} $ уже будет являться группой. Причем необязательно чтобы $ g $ и $ g^{-1} $ отличались. \\[2pt]
А так как мы имеем бесконечное число элементов, значит и бесконечное число пар элементов, то у нас как минимум бесконечно число подгрупп.












\end{document}
