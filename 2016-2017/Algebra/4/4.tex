\documentclass[a4paper,11pt]{article}
\usepackage[T1]{fontenc}
\usepackage[utf8]{inputenc}
\usepackage{graphicx}
\usepackage{xcolor}


\usepackage{amsmath,amssymb,amsthm,textcomp}
\usepackage{enumerate}
\usepackage{multicol}
\usepackage{tikz}
\usepackage[english, russian]{babel}

\usepackage{geometry}
\geometry{total={210mm,297mm},
	left=25mm,right=25mm,%
	bindingoffset=0mm, top=20mm,bottom=20mm}


% custom footers and headers
\usepackage{fancyhdr}
\pagestyle{fancy}
\lhead{}
\chead{}
\cfoot{}
\renewcommand{\headrulewidth}{0pt}
\renewcommand{\footrulewidth}{0pt}

%%%----------%%%----------%%%----------%%%----------%%%




\begin{document}
\textbf{\large Задача 1}
\medskip\hrule\medskip
\textit{Пусть $ G - $ группа всех диагональных матриц в $ GL_3(\mathbb{R}) $ и $ X = \mathbb{R}^3 $. Опишите все орбиты и все стабилизаторы для действия группы $ G $ на множетсве $ X $, заданного формулой $ (g, x) \to g \cdot x $. } \\ \\
Пусть $ x \in X \Rightarrow orb[x] = \{g \cdot x \; | \; g \in G \}, \; st[x] = \{g \in G \; | \; g \cdot x = x \} $  \\
И для орбиты и для стабилизатора ключевую роль играет произведение $ g \cdot x $:
\begin{gather*}
g = 
\begin{pmatrix}
a & 0 & 0 \\[2pt]
0 & b & 0 \\[2pt]
0 & 0 & c
\end{pmatrix}
\quad x = 
\begin{pmatrix}
k \\[2pt] 
l \\[2pt]
m
\end{pmatrix}
\text{ где } a, b, c \; \in \; \mathbb{R} / \{0\} \quad k, l, m \; \in \; \mathbb{R}  \\[2pt]
g \cdot x = 
\begin{pmatrix}
a & 0 & 0 \\[2pt]
0 & b & 0 \\[2pt]
0 & 0 & c
\end{pmatrix}
\cdot 
\begin{pmatrix}
k \\[2pt] 
l \\[2pt]
m
\end{pmatrix} 
= 
\begin{pmatrix}
ak \\[2pt] 
bl \\[2pt]
cm
\end{pmatrix} 
\end{gather*}
Таким образом наши координаты в исходном векторе умножаются на ненулевые числа, и мы получаем что орбита по просту совпадает с $ \mathbb{R}^3 $. \\[2pt]
Если же $ g - $ стабилизатор, тогда:
\begin{gather*}
g \cdot x = 
\begin{pmatrix}
ak \\[2pt] 
bl \\[2pt]
cm
\end{pmatrix}  = 
\begin{pmatrix}
k \\[2pt] 
l \\[2pt]
m
\end{pmatrix} \Rightarrow
\begin{cases}
ak = k \Rightarrow a = \begin{cases} a \; \in \; \mathbb{R}, \; k = 0 \\ 1, \; k \neq 0 \end{cases} \\[2pt]
bl = l \Rightarrow b = \begin{cases} b \; \in \; \mathbb{R}, \; l = 0 \\ 1, \; l \neq 0 \end{cases} \\[2pt]
cm = m \Rightarrow c = \begin{cases} c \; \in \; \mathbb{R}, \; m = 0 \\ 1, \; m \neq 0 \end{cases}
\end{cases}
\end{gather*}
\\ \\ \\






%%%----------%%%----------%%%----------%%%----------%%%


\textbf{\large Задача 2}
\medskip\hrule\medskip
\textit{Пусть $ G - $ группа всех верхнетреугольных матриц в $ SL_2(\mathbb{R}) $. Опишите все классы сопряженности в группе $ G $.} \\ \\
Так как $ det = 1 $ для любого объекта в $ SL_2(\mathbb{R}) $, то любой элемент группы принимает вид вид:
\begin{gather*}
\begin{pmatrix}
a & b \\[2pt]
0 & \frac{1}{a}
\end{pmatrix}
\end{gather*}
Рассмотрим произвольный объект $ s $  класса сопряженности элемента $ g: $ $ s = hgh^{-1} $:
\begin{gather*}
s = hgh^{-1} = 
\begin{pmatrix}
a & b \\[2pt]
0 & \frac{1}{a}
\end{pmatrix}
\begin{pmatrix}
m & k \\[2pt]
0 & \frac{1}{m}
\end{pmatrix}
\begin{pmatrix}
m & \frac{-abm^2 + a^2km + ab}{m} \\[2pt]
0 & \frac1{m}
\end{pmatrix} = 
\begin{pmatrix}
m & ab(m - \frac1{m}) + a^2k \\[2pt]
0 & \frac1{m}
\end{pmatrix}
\end{gather*}
Отсюда вытекает несколько вариантов:
\begin{itemize}
	\item $ m - \frac1{m} \neq 0, k \neq 0 \Rightarrow \text{матрица верхнетреугольная}$
	\item $ m - \frac1{m} \neq 0, k = 0 \Rightarrow \text{матрица верхнетреугольная}$
	\item $ m - \frac1{m} = 0 \; (m = \pm 1), k \neq 0 \Rightarrow \text{матрица верхнетреугольная, знак $ m[2][2] $ совпадает с $ k $}$
	\item $ m - \frac1{m} = 0 \; (m = \pm 1), k = 0 \Rightarrow \text{матрица диагональная}$
\end{itemize}
\newpage






%%%----------%%%----------%%%----------%%%----------%%%


\textbf{\large Задача 3}
\medskip\hrule\medskip
\textit{Для действия группы $ S_4 $ на себе сопряжениями найдите стабилизатор подстановки $ (1, 2, 3, 4) $.} \\ \\
Введем обозначения $ \sigma = (1, 2, 3, 4) = \begin{pmatrix} 1 & 2 & 3 & 4 \\[2pt]
 2 & 3 & 4 & 1\end{pmatrix} $. \\[2pt]
Стабилизатор $ st[\sigma] = \{s \in S_4 \; | \; s \cdot \sigma \cdot s^{-1} = \sigma \}$.
\begin{gather*}
s \cdot \sigma \cdot s^{-1} = \sigma \Longleftrightarrow s \cdot \sigma = \sigma \cdot s, \text{ откуда прямо следует} 
\end{gather*}
\begin{itemize}
	\item $ s(\sigma(1)) = \sigma(s(1)) \Rightarrow s(2) = \sigma(s(1))$
	\item $ s(\sigma(2)) = \sigma(s(2)) \Rightarrow s(3) = \sigma(s(2))$
	\item $ s(\sigma(3)) = \sigma(s(3)) \Rightarrow s(4) = \sigma(s(3))$
	\item $ s(\sigma(4)) = \sigma(s(4)) \Rightarrow s(1) = \sigma(s(4))$
\end{itemize}
На данном этапе довольно очевидно, что задавая $ s(1) $ все остальные $ s(i) $ определяются однозначно. 
\begin{itemize}
	\item $ s(1) = 1 
	\begin{cases}
		s(2) = \sigma(s(1)) = 2 \\[2pt]
		s(3) = \sigma(s(2)) = 3 \\[2pt]
		s(4) = \sigma(s(3)) = 4
	\end{cases} $
	\item $ s(1) = 2 
	\begin{cases}
		s(2) = \sigma(s(1)) = 3 \\[2pt]
		s(3) = \sigma(s(2)) = 4 \\[2pt]
		s(4) = \sigma(s(3)) = 1
	\end{cases} $
	\item $ s(1) = 3
	\begin{cases}
		s(2) = \sigma(s(1)) = 4 \\[2pt]
		s(3) = \sigma(s(2)) = 1 \\[2pt]
		s(4) = \sigma(s(3)) = 2
	\end{cases} $
	\item $ s(1) = 4 
	\begin{cases}
		s(2) = \sigma(s(1)) = 1 \\[2pt]
		s(3) = \sigma(s(2)) = 2 \\[2pt]
		s(4) = \sigma(s(3)) = 3
	\end{cases} $
\end{itemize}
Получаем, что $ st[\sigma] = \{
\begin{pmatrix}
1 & 2 & 3 & 4 \\
1 & 2 & 3 & 4
\end{pmatrix}, 
\begin{pmatrix}
1 & 2 & 3 & 4 \\
2 & 3 & 4 & 1
\end{pmatrix}
\begin{pmatrix}
1 & 2 & 3 & 4 \\
3 & 4 & 1 & 2
\end{pmatrix}
\begin{pmatrix}
1 & 2 & 3 & 4 \\
4 & 1 & 2 & 3
\end{pmatrix}
\} $
\\ \\ \\





%%%----------%%%----------%%%----------%%%----------%%%


\textbf{\large Задача 4}
\medskip\hrule\medskip
\textit{Пусть $ k, l \in \mathbb{N} $ и $ n = kl $. Реализуем группу $ \mathbb{Z}_k \times \mathbb{Z}_l $ как подгруппу в $ S_n $, используя доказательство теоремы Кели. Найдите необходимое и достаточное условие на числа $ k $ и $ l $, при которых эта подгруппа содержится в $ A_n $.} \\ \\
Запишем все элементы группы $ \mathbb{Z}_k \times \mathbb{Z}_l $ в табличку размера $ k \times l $ и рассмотрим возможную операцию. При этом согласно теореме Кели, мы можем каждому элементу $ (i, j) $ поставить в соответствие число $ 0 \leqslant il + j < n $ которому одназначно ставиться в соответствие подстановка. 
\begin{gather*}
\begin{vmatrix}
(0, 1) & \dots & (0, l - 1) \\[2pt]
\vdots & \vdots & \vdots \\[2pt]
(k - 1, 0) & \dots & (k - 1, l - 1)
\end{vmatrix}
\end{gather*}
Любой элемент $ с $ группы $ G $ представим в виде $ (i, j) = i(1, 0) + j(0, 1) $. При сложение  элемента $ (0, 1) $ с элементом $ (i, j) $, получаем что координата $ j $ сдвигается по циклу, а $ i $ остается тем же. Получаем, что четность такой подстановки равна $ k(l - 1) $, так как мы имеем  $ k $ циклов длины $ l $. Для элемента $ (1, 0) $ четность подстановки равна $ l^2(k - 1) $. Можно представить каждую строчку в виде блока и расположить их в порядке возрастания, тогда при прибавление $ (0, 1) $ все столбцы сместяться на 1. Получится, что последний столбец перейдет в 1 и станет образовывать инверсию со всеми членами до него, а таких ровно $ l * l * (k - 1) $, где  $ l $ - количество элементов в каждой строке, $ l $ - количство элементов в первой строке, $ (k - 1) $ - количество строк, с которыми образется инверсия. Так как все подстановки должны быть четными, то оба рассмотренных элемента должны не изменять четность, то есть:
\begin{gather*}
k(l - 1) \vdots 2 \qquad l^2(k - 1) \vdots 2
\end{gather*}
Откуда следует, что $ k $ и $ l $ одной четности. Мы получили необходимое и одновременно достаточное условие, так как любой элемент $ (i, j) $ единственным образом представляется как $ i(1, 0) + j(0, 1) $. 



\end{document}





