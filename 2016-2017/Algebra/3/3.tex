\documentclass[a4paper,11pt]{article}
\usepackage[T1]{fontenc}
\usepackage[utf8]{inputenc}
\usepackage{graphicx}
\usepackage{xcolor}


\usepackage{amsmath,amssymb,amsthm,textcomp}
\usepackage{enumerate}
\usepackage{multicol}
\usepackage{tikz}
\usepackage[english, russian]{babel}

\usepackage{geometry}
\geometry{total={210mm,297mm},
	left=25mm,right=25mm,%
	bindingoffset=0mm, top=20mm,bottom=20mm}


% custom footers and headers
\usepackage{fancyhdr}
\pagestyle{fancy}
\lhead{}
\chead{}
\cfoot{}
\renewcommand{\headrulewidth}{0pt}
\renewcommand{\footrulewidth}{0pt}

%%%----------%%%----------%%%----------%%%----------%%%

\begin{document}
\textbf{\large Задача 1}
\medskip\hrule\medskip
\textit{Сколько элементов порядков 2, 3, 4 и 6 в группе $ \mathbb{Z}_3 \times \mathbb{Z}_4 \times \mathbb{Z}_6 $?} \\ \\
Решая уравнение $ x^k = 0 $ мы получаем все $ x $, степень которых делит $ k $.  Для нашего удобства представим данные в удобной табличке:
\begin{center}
\begin{tabular}{|c|c|c|c|c|}
	\hline
	Степень & $ \mathbb{Z}_3 $ & $ \mathbb{Z}_4 $ & $ \mathbb{Z}_6 $ & $ \mathbb{Z}_3 \times \mathbb{Z}_4 \times \mathbb{Z}_6 $ \\
	\hline
	1 & 0 & 0 & 0 & 1 шт \\
	\hline
	2 & 0 & 0, 2 & 0, 3 & $ 1 1 \times 2 \times 2 - 1 = 3 $ шт \\
	\hline
	3 & 0, 1, 2 & 0 & 0, 2, 4 & $ 3 \times 1 \times 3 - 1 = 8 $ шт \\
	\hline
	4 & 0 & 0, 1, 2, 3 & 0, 3 & $ 1 \times 4 \times 2 - 1 - 3 = 4 $ шт \\
	\hline
	6 & 0, 1, 2 & 0, 2 & 0, 1, 2, 3, 4, 5& $ 3 \times 2 \times 6 - 1 - 3 - 8 = 24 $ шт \\
	\hline
\end{tabular}
\end{center}
Таким образом, получаем ответ 24. \\ \\ \\





%%%----------%%%----------%%%----------%%%----------%%%


\textbf{\large Задача 2}
\medskip\hrule\medskip
\textit{Сколько подгрупп порядков 3 и 15 в нециклической абелевой групппе порядка 45? }	\\ \\ 
$ 45 = 5 \cdot 3 \cdot 3 $. Это значит, что существуют только 2, с точностью до изоморфизма группы $ \mathbb{Z}_{45} $ и $ \mathbb{Z}_3 \times \mathbb{Z}_{15} $. Первая не подходит так как циклическая, остается вторая. По теореме о разложении в сумму примарных циклических групп, данная группа изоморфна $ \mathbb{Z}_5 \times \mathbb{Z}_{3} \times \mathbb{Z}_{3} $.   \\[2pt]
Так как число 3 - простое, то каждый (ненулевой элемент) порядка 3 пораждает группу порядка 3. Однако группы не должны пересекаться, а это значит у них не должно быть общих порождающих элементов. Аналогичо первой задаче, элементов порядка 3 у нас $ 1 \times 3 \times 3 - 1 = 8 $. Так как в любой подгруппе порядка 3 есть ровно 2 элемента порядка 3, то итоговый ответ: $ 8 / 2 = 4 $.  \\[2pt]
Любая группа порядка 15 изоморфна произведению групп порядка 3 и 5. В группе $ \mathbb{Z}_5 $ ровно 4 элемента порядка 5, в группах $ \mathbb{Z}_3 $ ровно по одному. Аналогично предыдущим высказываниям, различных элементов в группе получаем $ (5 \times 1 \times 1 - 1) / 4 = 1 $. В итоге, имеем количество различных групп порядка 10: $ 1 \times 4 = 4 $. \\ \\ \\





%%%----------%%%----------%%%----------%%%----------%%%


\textbf{\large Задача 3}
\medskip\hrule\medskip
\textit{Найдите в группе $ \mathbb{G} = \mathbb{Z} \times \mathbb{Z} $ подгруппу $ H $, для которой $ G \setminus H \simeq \mathbb{Z}_{10} \times \mathbb{Z}_{12} \times \mathbb{Z}_{15} $.} \\
\begin{gather*}
\mathbb{Z}_{mn} = \mathbb{Z}_{m} \times \mathbb{Z}_{n} \simeq (n, m) = 1
\end{gather*}
Отсюда получаем: $ \mathbb{Z}_{10} \times \mathbb{Z}_{12} \times \mathbb{Z}_{15} = (\mathbb{Z}_{2} \times \mathbb{Z}_{5}) \times (\mathbb{Z}_{3} \times \mathbb{Z}_{4}) \times (\mathbb{Z}_{3} \times \mathbb{Z}_{5}) \simeq (\mathbb{Z}_{2} \times \mathbb{Z}_{3} \times \mathbb{Z}_{5}) \times (\mathbb{Z}_{3} \times \mathbb{Z}_{4} \times \mathbb{Z}_{5}) \equiv \mathbb{Z}_{30} \times \mathbb{Z}_{60} $.   \\[2pt]
$ H = H_1 \times H_2 $. \\[2pt]
Из курса лекций знаем, что если $ G = (\mathbb{Z}, +) $ и $ H = n\mathbb{Z} $, то  $ G \setminus H = (\mathbb{Z}_n, +)$. \\[2pt]
Возьмем $ H_1 = 30\mathbb{Z}, H_2 = 60\mathbb{Z} $. При этом $ H_1 $ и $ H_2 $ нормальны в $ \mathbb{Z}, \mathbb{Z} \setminus H_1 \simeq \mathbb{Z}_{30}, \mathbb{Z} \setminus H_1 \simeq \mathbb{Z}_{60} $.
Собирая все фрагменты получаем:
\begin{gather*}
(\mathbb{Z} \times \mathbb{Z}) \setminus H \simeq  \mathbb{Z} \setminus H_1 \times \mathbb{Z} \setminus H_2 \simeq \mathbb{Z}_{30} \times \mathbb{Z}_{60} \simeq \mathbb{Z}_{10} \times \mathbb{Z}_{12} \times \mathbb{Z}_{15}
\end{gather*}
\newpage





%%%----------%%%----------%%%----------%%%----------%%%


\textbf{\large Задача 4}
\medskip\hrule\medskip
\textit{Пусть порядок конечной абелевой группы $ A $ делится на $ m $. Докажите, что в $ A $ есть подгруппа порядка $ m $.} \\ \\ 
Доказательство будет основано на индукции. База. Если $ G $ - циклическая группа порядка $ N $, и $ f $
ее образующая, то циклическая группа порожденная элементом $ f^{\frac{N}{n}} $, имеет порядок $ n $ для любого делителя $ n $ числа $ N $.   \\[2pt]
Пусть $ G $ не является цикилической. Это значит, что она изоморфна прямому произведению абелевых групп меньших размеров: $ G \simeq G_1 \times G_2 $, где $ |G_1| = N_1, \; |G_2| = N_2 $. При этом $ |G| = N_1 \times N_2 $ Тогда найдутся такие числа $ n_1 | N_1, \; n_2 | N_2 $, что $ n = n_1n_2 $. Тогда по индукционному предположению существуют группы $ H_1 \leq G_1, \; H_2 \leq G_2 $, такие что $ |H_1| = n_1, \; |H_2| = n_2 $. Получаем $ H_1 \times H_2 \leq G , \; |H_1 \times H_2| = n_1 \times n_2 = n$, что и требовалось доказать. 









\end{document}