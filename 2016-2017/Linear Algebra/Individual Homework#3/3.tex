\documentclass[10pt]{article}
\usepackage[utf8]{inputenc}
\usepackage{float}
\usepackage{amsmath}
\usepackage{amsfonts}
\usepackage{amssymb}
\usepackage{graphicx}
\usepackage{color}
\usepackage[english, russian]{babel}
\title{}
\date{Егерев Артем, БПМИ-167}
\author{Домашняя работа по дискретной математике 17}
\setlength{\textheight}{40\baselineskip}
\setlength{\textheight}{\baselinestretch\textheight}
\addtolength{\textheight}{\topskip}

\textheight=24cm
\textwidth=16cm
\oddsidemargin=0pt
\topmargin=-1.5cm
\parindent=24pt
\parskip=0pt
\tolerance=2000
\flushbottom

\begin{document}
\begin{flushright}
	Егерев Артем \\
	Индивидуальное домашнее задание №3 \\
	Вариант 4
\end{flushright}
\textbf{\Large Задача 1}
\medskip\hrule\medskip
\textsl{В пространсте $\mathbb{R}^5$ даны векторы:}
\begin{align*}
v_1 &= (-26, 1, 7, -47, 4), v_2 =(-13, -4, -1, 35, 2) \\ v_3 &=(-29, -9, 4, 16, -5), v_4 =(42, 6, -2, -40, -9)
\end{align*}
\textsl{(a) Найти базис подпространства $\langle v_1, v_2, v_3, v_4 \rangle$ \\
(б) Дополните полученный в пункте а) базис до базиса всего пространства $\mathbb{R}^5$}  \\

\fbox{\textit{Решение:}} \\ \\
 \textsl{(a)} Базис есть максимальный набор линейно независимых векторов. Для того чтобы привести вектора $v_1, ... , v_4$ к линейно независимым, составим матрицу из векторов и приведем ее к ступенчатому виду:
\begin{figure}[H]
	\raggedright
	$\begin{pmatrix}
		v_1 \\
		v_2 \\
		v_3 \\
		v_4
	\end{pmatrix}$
	\text{ = }
	$\begin{pmatrix}
		-26 & 1 & 7 & -47 & 4 \\
		-13 & -4 & -1 & 35 & 2 \\
		-29 & -9 & 4 & 16 & -5 \\
		42 & 6 & -2 & -40 & -9 
	\end{pmatrix}$
	\text{ = }
	$\begin{pmatrix}
		-26 & 1 & 7 & -47 & 4 \\
		-13 & -4 & -1 & 35 & 2 \\
		-29 & -9 & 4 & 16 & -5 \\
		0 & -7 & 1 & 11 & -12 
	\end{pmatrix}$
	\text{ = }

	\text{ = }
	$\begin{pmatrix}
		0 & 9 & 9 & -117 & 0 \\
		-13 & -4 & -1 & 35 & 2 \\
		-29 & -9 & 4 & 16 & -5 \\
		0 & -7 & 1 & 11 & -12 
	\end{pmatrix}$
	\text{ = }
	$\begin{pmatrix}
		-13 & -4 & -1 & 35 & 2 \\
		-29 & -9 & 4 & 16 & -5 \\
		0 & 9 & 9 & -117 & 0 \\
		0 & -7 & 1 & 11 & -12 
	\end{pmatrix}$
	\text{ = }	
	
	\text{ = }
	$\begin{pmatrix}
		-13 & -4 & -1 & 35 & 2 \\
		-3 & -1 & 6 & -54 & -9 \\
		0 & 9 & 9 & -117 & 0 \\
		0 & -7 & 1 & 11 & -12
	\end{pmatrix}$
	\text{ = }
	$\begin{pmatrix}
		-1 & 0 & -25 & 251 & 38 \\
		-3 & -1 & 6 & -54 & -9 \\
		0 & 9 & 9 & -117 & 0 \\
		0 & -7 & 1 & 11 & -12
	\end{pmatrix}$
	\text{ = }	

	\text{ = }
	$\begin{pmatrix}
		-1 & 0 & -25 & 251 & 38 \\
		0 & -1 & 81 & -807 & -123 \\
		0 & 9 & 9 & -117 & 0 \\
		0 & -7 & 1 & 11 & -12
	\end{pmatrix}$
	\text{ = }
	$\begin{pmatrix}
		-1 & 0 & -25 & 251 & 38 \\
		0 & -1 & 81 & -807 & -123 \\
		0 & 0 & 738 & -7380 & -1170 \\
		0 & -7 & 1 & 11 & -12
	\end{pmatrix}$
	\text{ = }	

	\text{ = }
	$\begin{pmatrix}
		-1 & 0 & -25 & 251 & 38 \\
		0 & -1 & 81 & -807 & -123 \\
		0 & 0 & 738 & -7380 & -1170 \\
		0 & 0 & -566 & 5660 & 849
	\end{pmatrix}$
	\text{ = }
	$\begin{pmatrix}
		-1 & 0 & -25 & 251 & 38 \\
		0 & -1 & 81 & -807 & -123 \\
		0 & 0 & 172 & -1720 & -258 \\
		0 & 0 & -566 & 5660 & 849
	\end{pmatrix}$
	\text{ = }
	
	\text{ = }
	$\begin{pmatrix}
		-1 & 0 & -25 & 251 & 38 \\
		0 & -1 & 81 & -807 & -123 \\
		0 & 0 & 172 & -1720 & -258 \\
		0 & 0 & -50 & 500 & 75
	\end{pmatrix}$
	\text{ = }
	$\begin{pmatrix}
		-1 & 0 & -25 & 251 & 38 \\
		0 & -1 & 81 & -807 & -123 \\
		0 & 0 & 22 & -220 & -33 \\
		0 & 0 & -50 & 500 & 75
	\end{pmatrix}$
	\text{ = }
\end{figure}
\begin{figure}[H]	
	\text{ = }
	$\begin{pmatrix}
		-1 & 0 & -25 & 251 & 38 \\
		0 & -1 & 81 & -807 & -123 \\
		0 & 0 & 22 & -220 & -33 \\
		0 & 0 & -6 & 60 & 9 \\
		0 & 0 & 0 & 0 & 0
	\end{pmatrix}$
	\text{ = }
	$\begin{pmatrix}
		-1 & 0 & -25 & 251 & 38 \\
		0 & -1 & 81 & -807 & -123 \\
		0 & 0 & -2 & 20 & 3 \\
		0 & 0 & -6 & 60 & 9
	\end{pmatrix}$
	\text{ = }	
			
	\text{ = }
	$\begin{pmatrix}
		-1 & 0 & -25 & 251 & 38 \\
		0 & -1 & 81 & -807 & -123 \\
		0 & 0 & -2 & 20 & 3 \\
	\end{pmatrix}$
	\text{ = }
	$\begin{pmatrix}
		-1 & 0 & -25 & 251 & 38 \\
		0 & -1 & 1 & -7 & -3 \\
		0 & 0 & -2 & 20 & 3 \\
	\end{pmatrix}$
	\text{ = }
	
	\text{ = }
	$\begin{pmatrix}
		-1 & 0 & -1 & 11 & 2 \\
		0 & -1 & 1 & -7 & -3 \\
		0 & 0 & -2 & 20 & 3 \\
	\end{pmatrix}$
	\text{ = }
	$\begin{pmatrix}
		-1 & 0 & 1 & -9 & -1 \\
		0 & -1 & 1 & -7 & -3 \\
		0 & 0 & -2 & 20 & 3 \\
	\end{pmatrix}$	
\end{figure}
На данном этапе уже очевидно, что данные вектора являются линейно независимыми. \\ \textsl{(б)}

\begin{center}
	\begin{tabular}{ | r | r | r | r | r | r |}
		\hline
		$s_1$ & -1 & 0 & 1 & -9 & -1 \\ \hline
		$s_2$ & 0 & -1 & 1 & -7 & -3 \\ \hline
		$s_3$ & 0 & 0 & -2 & 20 & 3 \\ \hline
		$s_4$ & 0 & 0 & 0 & 1 & 0 \\ \hline
		$s_5$ & 0 & 0 & 0 & 0 & 1 \\ \hline
	\end{tabular}
\end{center}
Добавив вектора $s_4 \text{ и } s_5$ мы получили набор из 5 линейно независимых векторов в пространстве размерности 5 $\Rightarrow$ это и будет базис.
\newpage












\textbf{\Large Задача 2}
\medskip\hrule\medskip 
\textsl{Пусть $U$ - подпространство в $\mathbb{R}^4$, натянутое на вектора: }
\begin{align*}
	u_1 &= (-34, -26, -152, 10), u_2 = (-4, 7, -38, -29) \\
	u_3 &= (-16, -19, -58, 25), u_4 = (-3, -5, -8, 9)
\end{align*}
\textsl{Составьте однородную систему линейных уравнений, у которой множество решений совпадает с $U$.} \\ 

\fbox{\textit{Решение:}} \\ \\
 Для начала найдем базис в $U$, аналогично тому как мы это делали в \textit{Задаче 1}.
\begin{figure}[H] 
\raggedright
$\begin{pmatrix}
u_1 \\
u_2 \\
u_3 \\
u_4
\end{pmatrix}$
\text{ = }
$\begin{pmatrix}
	-34 & -26 & -152 & 10 \\
	-4 & 7 & -38 & -29  \\
	-16 & -19 & -58 & 25 \\
	-3 & -5 & -8 & 9
\end{pmatrix}$
\text{ = }
$\begin{pmatrix}
	-2 & 12 & -36 & -40 \\
	-4 & 7 & -38 & -29  \\
	-16 & -19 & -58 & 25 \\
	-3 & -5 & -8 & 9
\end{pmatrix}$
\text{ = }
\end{figure}
\begin{figure}[H]
\text{ = }
$\begin{pmatrix}
	-2 & 12 & -36 & -40 \\
	-4 & 7 & -38 & -29  \\
	0 & -47 & 94 & 141 \\
	-3 & -5 & -8 & 9
\end{pmatrix}$
\text{ = }
$\begin{pmatrix}
	-2 & 12 & -36 & -40 \\
	0 & -17 & 34 & 51  \\
	0 & -47 & 94 & 141 \\
	-3 & -5 & -8 & 9
\end{pmatrix}$
\text{ = }

\text{ = }
$\begin{pmatrix}
	1 & 17 & -28 & -49 \\
	0 & -17 & 34 & 51  \\
	0 & -47 & 94 & 141 \\
	-3 & -5 & -8 & 9
\end{pmatrix}$
\text{ = }
$\begin{pmatrix}
	1 & 17 & -28 & -49 \\
	0 & -17 & 34 & 51  \\
	0 & -47 & 94 & 141 \\
	0 & 46 & -92 & -138
\end{pmatrix}$
\text{ = }

\text{ = }
$\begin{pmatrix}
1 & 17 & -28 & -49 \\
0 & -17 & 34 & 51  \\
0 & -1 & 2 & 3 \\
0 & 46 & -92 & -138
\end{pmatrix}$
\text{ = }
$\begin{pmatrix}
1 & 17 & -28 & -49 \\
0 & 0 & 0 & 0 &  \\
0 & -1 & 2 & 3 \\
0 & 46 & -92 & -138
\end{pmatrix}$
\text{ = }

\text{ = }
$\begin{pmatrix}
1 & 17 & -28 & -49 \\
0 & -1 & 2 & 3 \\
0 & 0 & 0 & 0
\end{pmatrix}$
\text{ = }
$\begin{pmatrix}
1 & 0 & 6 & 2 \\
0 & -1 & 2 & 3 \\
\end{pmatrix}$ 
\end{figure}
Из курса по линейной алгебре мы знаем, что искомые коэффиценами в однородной системе уравнений, задающей $U$, является ФСР матрицы, в строки  которой записан непоредственно базис в самом $U$:
\[
\bordermatrix{
	& x_1 & x_2 & x_3 & x_4 \cr 
	& 1 & 0 & 6 & 2 \cr
	& 0 & -1 & 2 & 3 \cr
}
\]
В качестве свободных переменных возьмем $x_3$ и $x_4$, отсюда ФСР сразу принимает  вид $(-2, 3, 0, 1), (-6, 2, 1, 0)$. Теперь без труда можем выписать нашу полученную систему:
\begin{equation*}
\begin{cases}
-2 x_1 + 3x_2 + x_4 = 0 \\
-6 x_1 + 2 x_2 + x_3 = 0 \\
\end{cases}
\end{equation*}
\newpage










\textbf{\Large Задача 3}
\medskip\hrule\medskip 
\textsl{Найти базис и размерность каждого из подпространст $L_1, L_2, U = L_1 + L_2, W = L_1 \cap L_2$ пространства $\mathbb{R}^4$, если $L_1$ - линейная оболочка векторов:}
\begin{align*}
a_1 &= (-4, -6, -7, 0), \quad a_2 = (0, -1, -3, -4) \\
a_3 &= (-8, -11, -11, 4), \quad a_4 = (-4, -4, -1, 8)
\end{align*}
\textsl{а $L_2$ - линейная оболочка векторов:}
\begin{align*}
b_1 &= (8, 12, 14, 0), \quad b_2 = (8, 11, 11, -4) \\
b_3 &= (8, 13, 17, 4), \quad b_4 = (-8, -10, -8, 8)
\end{align*} \\ \\

\fbox{\textit{Решение:}} \\ \\
Перейдем от линейных оболочек непосредственно к базисам:
\begin{figure}[H] 
 $\begin{pmatrix}
 a_1 \\
 a_2 \\
 a_3 \\
 a_4
 \end{pmatrix}$
 \text{ = }
 $\begin{pmatrix}
-4 & -6 & -7 & 0 \\
0 & -1 & -3 & -4 \\
-8 & -11 & -11 & 4 \\
-4 & -4 & -1 & 8 \\
 \end{pmatrix}$
 \text{ = }
 $\begin{pmatrix}
 -4 & -6 & -7 & 0 \\
 0 & -1 & -3 & -4 \\
 0 & 1 & 3 & 4 \\
 -4 & -4 & -1 & 8 \\
 \end{pmatrix}$
 \text{ = }
 
 \text{ = }
 $\begin{pmatrix}
 -4 & -6 & -7 & 0 \\
 0 & -1 & -3 & -4 \\
 0 & 1 & 3 & 4 \\
 0 & 2 & 6 & 8 \\
 \end{pmatrix}$
 \text{ = }
 $\begin{pmatrix}
-4 & -6 & -7 & 0 \\
0 & 0 & 0 & 0 \\
0 & 1 & 3 & 4 \\
0 & 2 & 6 & 8 \\
 \end{pmatrix}$
 \text{ = }
 $\begin{pmatrix}
-4 & -6 & -7 & 0 \\
0 & 0 & 0 & 0 \\
0 & 1 & 3 & 4 \\
0 & 0 & 0 & 0 \\
 \end{pmatrix}$
 \text{ = }
 
 \text{ = }
 $\begin{pmatrix}
 -4 & -6 & -7 & 0 \\
 0 & 1 & 3 & 4 \\ 
 \end{pmatrix}$
\end{figure}
\begin{figure}[H] 
	$\begin{pmatrix}
	b_1 \\
	b_2 \\
	b_3 \\
	b_4
	\end{pmatrix}$
	\text{ = }
	$\begin{pmatrix}
	8 & 12 & 14 & 0 \\
	8 & 11 & 11 & -4 \\
	8 & 13 & 17 & 4 \\
	-8 & -10 & -8 & 8 \\
	\end{pmatrix}$
	\text{ = }
	$\begin{pmatrix}
	8 & 12 & 14 & 0 \\
	8 & 11 & 11 & -4 \\
	8 & 13 & 17 & 4 \\
	0 & 2 & 6 & 8 \\
	\end{pmatrix}$
	\text{ = }
	
	\text{ = }
	$\begin{pmatrix}
	8 & 12 & 14 & 0 \\
	8 & 11 & 11 & -4 \\
	0 & 1 & 3 & 4 \\
	0 & 2 & 6 & 8 \\
	\end{pmatrix}$
	\text{ = }
	$\begin{pmatrix}
	8 & 12 & 14 & 0 \\
	0 & -1 & -3 & -4 \\
	0 & 1 & 3 & 4 \\
	0 & 2 & 6 & 8 \\
	\end{pmatrix}$
	\text{ = }
	$\begin{pmatrix}
	8 & 12 & 14 & 0 \\
	0 & -1 & -3 & -4 \\
	0 & 1 & 3 & 4 \\
	0 & 0 & 0 & 0 \\
	\end{pmatrix}$
	\text{ = }
	
	\text{ = }
	$\begin{pmatrix}
	8 & 12 & 14 & 0 \\
	0 & -1 & -3 & -4 \\ 
	\end{pmatrix}$
\end{figure}
Базис суммы  ($U$) - есть базис линейной оболочки базисов пространств $L_1$ и $L_2$:
\begin{figure}[H]
$\begin{pmatrix}
a_1' \\
a_2' \\
b_1' \\
b_2'
\end{pmatrix}$
\text{ = }
$\begin{pmatrix}
-4 & -6 & -7 & 0 \\
0 & 1 & 3 & 4 \\
8 & 12 & 14 & 0 \\
0 & -1 & -3 & -4 \\
\end{pmatrix}$
\text{ = }
$\begin{pmatrix}
-4 & -6 & -7 & 0 \\
0 & 1 & 3 & 4 \\
0 & 0 & 0 & 0 \\
0 & -1 & -3 & -4 \\
\end{pmatrix}$
\text{ = }

\text{ = }
$\begin{pmatrix}
-4 & -6 & -7 & 0 \\
0 & 1 & 3 & 4 \\
0 & 0 & 0 & 0 \\
0 & 0 & 0 & 0 \\
\end{pmatrix}$
\text{ = }
$\begin{pmatrix}
-4 & -6 & -7 & 0 \\
0 & 1 & 3 & 4 \\
\end{pmatrix}$
\end{figure}
$\Rightarrow$ нашли базис в $U$, размерность =  2 \\
Возьмем произвольный вектор $p$, входящий в пересечение оболочек, тогда:
$$
p = \alpha_1 a_1' + \alpha_2 a_2' = \beta_1 b_1' + \beta_2 b_2'
$$
Отсюда  любая нетривиальная комбинация коэффицентов: $\alpha_1 a_1' + \alpha_2 a_2' - \beta_1 b_1' - \beta_2 b_2' = 0$ будет давать нам ненулевой вектор  $p$.
\begin{figure}[H]
$\begin{pmatrix}
a_1',
a_2', 
b_1', 
b_2'
\end{pmatrix}$
\text{ = }
$\begin{pmatrix}
-4 & 0 & 8 & 0 \\
-6 & 1 & 12 & -1 \\
-7 & 3 & 14 & -3 \\
0 & 4 & 0 & -4 \\
\end{pmatrix}$
\text{ = }
$\begin{pmatrix}
-4 & 0 & 8 & 0 \\
1 & -2 & -2 & 2 \\
-7 & 3 & 14 & -3 \\
0 & 4 & 0 & -4 \\
\end{pmatrix}$
\text{ = }

\text{ = }
$\begin{pmatrix}
0 & -8 & 0 & 8 \\
1 & -2 & -2 & 2 \\
0 & -11 & 0 & 11 \\
0 & 4 & 0 & -4 \\
\end{pmatrix}$
\text{ = }
$\begin{pmatrix}
1 & -2 & -2 & 2 \\
0 & -8 & 0 & 8 \\
0 & -11 & 0 & 11 \\
0 & 4 & 0 & -4 \\
\end{pmatrix}$
\text{ = }
$\begin{pmatrix}
1 & -2 & -2 & 2 \\
0 & -1 & 0 & 1 \\
0 & 0 & 0 & 0 \\
0 & 0 & 0 & 0 \\
\end{pmatrix}$
\text{ = }

\text{ = }
$\begin{pmatrix}
1 & 0 & -2 & 0 \\
0 & -1 & 0 & 1 \\
\end{pmatrix}$
\end{figure}
Взяв за свободные переменные ($-\beta_1$) и ($-\beta_2$) ФСР данной матрицы будут вектора (2, 0, 1, 0) и (0, 1, 0, 1). Таким образом базис перечечения есть: $-1 \cdot b_1' + 0 \cdot b_2'$ и $0 \cdot b_1' + (-1)  \cdot b_2'$, или что тоже самое $(-8, -12, -14, 0)$ и $(0, 1, 3, 4)$ - размерность 2. \\ \\
\textit{Стоит отметить тот факт, что мы уже раньше, не вычисляя всего этого, могли дать ответ на данный пункт. Так как и $ L_1 $, и $ L_2 $, и $L_1 + L_2 $ имеют размер 2, то достаточно очевидно, что они просто совпадают, поэтому базис пересечения есть просто базис одного из пространств. }
\newpage














\textbf{\Large Задача 4}
\medskip\hrule\medskip
\textsl{В пространстве $\mathbb{R}^4$ рассмотрим подпространства $U = \langle v_1, v_2\rangle$ и $W = \langle v_3, v_4\rangle$, где }
\begin{align*}
	v_1 &= (10, -16, -10, 19), v_2 = (-2, 18, 19, -7) \\
	v_3 &= (15, -2, 10, 3), v_4 = (-2, 25, 24, -10)
\end{align*}
\textsl{(а) Докажите, что $\mathbb{R}^4 = U \oplus W$ \\ (б) Найдите проекцию вектора $x = (25б -18, 0, 22)$ на подпростанство  $W$ вдволь подпространства $U$} \\ 

\fbox{\textit{Решение:}} \\ \\
\textsl{(a)} Найдем базис в cумме подпространств $ U $ и $ W $:
\begin{figure}[H]
\raggedright
$\begin{pmatrix}
v_1 \\
v_2 \\
v_3 \\
v_4
\end{pmatrix}$
\text{ = }
$\begin{pmatrix}
10 & -16 & -10 & 19 \\
-2 & 18 & 19 & -7 \\
15 & -2 & 10 & 3 \\
-2 & 25 & 24 & -10 \\
\end{pmatrix}$
\text{ = }
$\begin{pmatrix}
10 & -16 & -10 & 19 \\
-2 & 18 & 19 & -7 \\
5 & 14 & 20 & -16 \\
-2 & 25 & 24 & -10 \\
\end{pmatrix}$
\text{ = }

\text{ = }
$\begin{pmatrix}
0 & -44 & -50 & 51 \\
-2 & 18 & 19 & -7 \\
5 & 14 & 20 & -16 \\
-2 & 25 & 24 & -10 \\
\end{pmatrix}$
\text{ = }
$\begin{pmatrix}
0 & -44 & -50 & 51 \\
-2 & 18 & 19 & -7 \\
5 & 14 & 20 & -16 \\
0 & 7 & 5 & -3 \\
\end{pmatrix}$
\text{ = }

\text{ = }
$\begin{pmatrix}
0 & -44 & -50 & 51 \\
-2 & 18 & 19 & -7 \\
1 & 50 & 58 & -30 \\
0 & 7 & 5 & -3 \\
\end{pmatrix}$
\text{ = }
$\begin{pmatrix}
0 & -44 & -50 & 51 \\
0 & 118 & 135 & -67 \\
1 & 50 & 58 & -30 \\
0 & 7 & 5 & -3 \\
\end{pmatrix}$
\text{ = }

\text{ = }
$\begin{pmatrix}
1 & 50 & 58 & -30 \\
0 & 118 & 135 & -67 \\
0 & -44 & -50 & 51 \\
0 & 7 & 5 & -3 \\
\end{pmatrix}$
\text{ = }
$\begin{pmatrix}
1 & 50 & 58 & -30 \\
0 & 30 & 35 & 35 \\
0 & -44 & -50 & 51 \\
0 & 7 & 5 & -3 \\
\end{pmatrix}$
\text{ = }

\text{ = }
$\begin{pmatrix}
1 & 50 & 58 & -30 \\
0 & 30 & 35 & 35 \\
0 & -14 & -15 & 86 \\
0 & 7 & 5 & -3 \\
\end{pmatrix}$
\text{ = }
$\begin{pmatrix}
1 & 50 & 58 & -30 \\
0 & 30 & 35 & 35 \\
0 & 0 & -5 & 80 \\
0 & 7 & 5 & -3 \\
\end{pmatrix}$
\text{ = }

\text{ = }
$\begin{pmatrix}
1 & 50 & 58 & -30 \\
0 & 2 & 15 & 47 \\
0 & 0 & -5 & 80 \\
0 & 7 & 5 & -3 \\
\end{pmatrix}$
\text{ = }
$\begin{pmatrix}
1 & 50 & 58 & -30 \\
0 & 2 & 15 & 47 \\
0 & 0 & -5 & 80 \\
0 & 1 & -40 & -144 \\
\end{pmatrix}$
\text{ = }

\text{ = }
$\begin{pmatrix}
1 & 50 & 58 & -30 \\
0 & 0 & 95 & 335 \\
0 & 0 & -5 & 80 \\
0 & 1 & -40 & -144 \\
\end{pmatrix}$
\text{ = }
$\begin{pmatrix}
1 & 50 & 58 & -30 \\
0 & 1 & -40 & -144 \\
0 & 0 & -5 & 80 \\
0 & 0 & 95 & 335 \\
\end{pmatrix}$
\text{ = }
\end{figure}
\begin{figure}[H]
\text{ = }
$\begin{pmatrix}
1 & 50 & 58 & -30 \\
0 & 1 & -40 & -144 \\
0 & 0 & -5 & 80 \\
0 & 0 & 0 & 1855 \\
\end{pmatrix}$
\end{figure}
4 линейно независимых вектора в пространстве размерности 4 $\Rightarrow U \oplus W = \mathbb{R}^4$. \\ \\
\textsl{(б)} Пусть $x = \alpha_1 v_1 + \alpha_2 v_2 + \alpha_3 v_3 + \alpha_4 v_4$, где $\alpha_1 \dots \alpha_4$ - какие-то неизвестные. Тогда: 
\[
\bordermatrix{
	& \alpha_1 & \alpha_2 & \alpha_3 & \alpha_4 & \cr
	& 10 & -2 & 15 & -2 & 25 \cr
	& -16 & 18 & -2 & 25 & -18 \cr
	& -10 & 19 & 10 & 24 & 0 \cr
	& 19 & -7 & 3 & -10 & 22 \cr
}
\text{  =}
\bordermatrix{
	& \alpha_1 & \alpha_2 & \alpha_3 & \alpha_4 & \cr
	& 10 & -2 & 15 & -2 & 25 \cr
	& -16 & 18 & -2 & 25 & -18 \cr
	& 0 & 17 & 25 & 22 & 25 \cr
	& 19 & -7 & 3 & -10 & 22 \cr
}
\text{  =}
\]

\[
\text{ = }
\bordermatrix{
	& \alpha_1 & \alpha_2 & \alpha_3 & \alpha_4 & \cr
	& 10 & -2 & 15 & -2 & 25 \cr
	& -16 & 18 & -2 & 25 & -18 \cr
	& 0 & 17 & 25 & 22 & 25 \cr
	& -1 & -3 & -27 & -6 & -28 \cr
}
\text{ =}
\bordermatrix{
& \alpha_1 & \alpha_2 & \alpha_3 & \alpha_4 & \cr
& 0 & -32 & -255 & -62 & -255 \cr
& -16 & 18 & -2 & 25 & -18 \cr
& 0 & 17 & 25 & 22 & 25 \cr
& -1 & -3 & -27 & -6 & -28 \cr
}
\]

\[
\text{ = }
\bordermatrix{
	& \alpha_1 & \alpha_2 & \alpha_3 & \alpha_4 & \cr
	& 0 & -32 & -255 & -62 & -255 \cr
	& 0 & 66 & 430 & 121 & 430 \cr
	& 0 & 17 & 25 & 22 & 25 \cr
	& -1 & -3 & -27 & -6 & -28 \cr
}
\text{  =}
\bordermatrix{
	& \alpha_1 & \alpha_2 & \alpha_3 & \alpha_4 & \cr
	& -1 & -3 & -27 & -6 & -28 \cr
	& 0 & 66 & 430 & 121 & 430 \cr
	& 0 & 17 & 25 & 22 & 25 \cr
	& 0 & -32 & -255 & -62 & -255 \cr
}
\]

\[
\text{ = }
\bordermatrix{
	& \alpha_1 & \alpha_2 & \alpha_3 & \alpha_4 & \cr
	& -1 & -3 & -27 & -6 & -28 \cr
	& 0 & 2 & -80 & -3 & -80 \cr
	& 0 & 17 & 25 & 22 & 25 \cr
	& 0 & -32 & -255 & -62 & -255 \cr
}
\text{  =}
\bordermatrix{
& \alpha_1 & \alpha_2 & \alpha_3 & \alpha_4 & \cr
& -1 & -3 & -27 & -6 & -28 \cr
& 0 & 2 & -80 & -3 & -80 \cr
& 0 & 17 & 25 & 22 & 25 \cr
& 0 & 2 & -205 & -18 & -205 \cr
}
\]

\[
\text{ = }
\bordermatrix{
	& \alpha_1 & \alpha_2 & \alpha_3 & \alpha_4 & \cr
	& -1 & -3 & -27 & -6 & -28 \cr
	& 0 & 2 & -80 & -3 & -80 \cr
	& 0 & 17 & 25 & 22 & 25 \cr
	& 0 & 0 & -125 & -15 & -125 \cr
}
\text{  =}
\bordermatrix{
	& \alpha_1 & \alpha_2 & \alpha_3 & \alpha_4 & \cr
	& -1 & -3 & -27 & -6 & -28 \cr
	& 0 & 2 & -80 & -3 & -80 \cr
	& 0 & -1 & 745 & 49 & 745 \cr
	& 0 & 0 & -125 & -15 & -125 \cr
}
\]

\[
\text{ = }
\bordermatrix{
	& \alpha_1 & \alpha_2 & \alpha_3 & \alpha_4 & \cr
	& -1 & -3 & -27 & -6 & -28 \cr
	& 0 & 0 & 1410 & 95 & 1410 \cr
	& 0 & -1 & 745 & 49 & 745 \cr
	& 0 & 0 & -125 & -15 & -125 \cr
}
\text{  =}
\bordermatrix{
	& \alpha_1 & \alpha_2 & \alpha_3 & \alpha_4 & \cr
	& -1 & -3 & -27 & -6 & -28 \cr
	& 0 & -1 & 745 & 49 & 745 \cr
	& 0 & 0 & 1410 & 95 & 1410 \cr
	& 0 & 0 & -125 & -15 & -125 \cr
}
\]

\[
\text{ = }
\bordermatrix{
	& \alpha_1 & \alpha_2 & \alpha_3 & \alpha_4 & \cr
	& -1 & -3 & -27 & -6 & -28 \cr
	& 0 & -1 & 745 & 49 & 745 \cr
	& 0 & 0 & -90 & -85 & -90 \cr
	& 0 & 0 & -125 & -15 & -125 \cr
}
\text{  =}
\bordermatrix{
	& \alpha_1 & \alpha_2 & \alpha_3 & \alpha_4 & \cr
	& -1 & -3 & -27 & -6 & -28 \cr
	& 0 & -1 & 745 & 49 & 745 \cr
	& 0 & 0 & -90 & -85 & -90 \cr
	& 0 & 0 & -35 & 70 & -35 \cr
}
\]

\[
\text{ = }
\bordermatrix{
	& \alpha_1 & \alpha_2 & \alpha_3 & \alpha_4 & \cr
	& -1 & -3 & -27 & -6 & -28 \cr
	& 0 & -1 & 745 & 49 & 745 \cr
	& 0 & 0 & 15 & -295 & 15 \cr
	& 0 & 0 & -35 & 70 & -35 \cr
}
\text{  =}
\bordermatrix{
	& \alpha_1 & \alpha_2 & \alpha_3 & \alpha_4 & \cr
	& -1 & -3 & -27 & -6 & -28 \cr
	& 0 & -1 & 745 & 49 & 745 \cr
	& 0 & 0 & 15 & -295 & 15 \cr
	& 0 & 0 & -5 & -520 & -5 \cr
}
\]

\[
\text{ = }
\bordermatrix{
	& \alpha_1 & \alpha_2 & \alpha_3 & \alpha_4 & \cr
	& -1 & -3 & -27 & -6 & -28 \cr
	& 0 & -1 & 745 & 49 & 745 \cr
	& 0 & 0 & 0 & -1855 & 0 \cr
	& 0 & 0 & -5 & -520 & -5 \cr
}
\text{  =}
\bordermatrix{
	& \alpha_1 & \alpha_2 & \alpha_3 & \alpha_4 & \cr
	& -1 & -3 & -27 & -6 & -28 \cr
	& 0 & -1 & 745 & 49 & 745 \cr
	& 0 & 0 & -5 & -520 & -5 \cr
	& 0 & 0 & 0 & -1855 & 0 \cr
}
\]

\[
\text{ = }
\bordermatrix{
	& \alpha_1 & \alpha_2 & \alpha_3 & \alpha_4 & \cr
	& -1 & -3 & -27 & -6 & -28 \cr
	& 0 & -1 & 0 & -77431 & 0 \cr
	& 0 & 0 & -5 & -520 & -5 \cr
	& 0 & 0 & 0 & -1855 & 0 \cr
}
\text{  =}
\bordermatrix{
	& \alpha_1 & \alpha_2 & \alpha_3 & \alpha_4 & \cr
	& -1 & 0 & -27 & 232287 & -28 \cr
	& 0 & -1 & 0 & -77431 & 0 \cr
	& 0 & 0 & -5 & -520 & -5 \cr
	& 0 & 0 & 0 & -1855 & 0 \cr
}
\]

\[
\text{ = }
\bordermatrix{
	& \alpha_1 & \alpha_2 & \alpha_3 & \alpha_4 & \cr
	& -1 & 0 & -2 & 234887 & -3 \cr
	& 0 & -1 & 0 & -77431 & 0 \cr
	& 0 & 0 & -5 & -520 & -5 \cr
	& 0 & 0 & 0 & -1855 & 0 \cr
}
\]
Отсюда $\alpha_4 = 0, \alpha_3 = 1, \alpha_2 = 0, \alpha_1 = 1 \Rightarrow $ проекция $x$ на $W$ равна $\alpha_3 v_3 + \alpha_4 v_4 = (15, -2, 10, 3)$. 

\end{document}
